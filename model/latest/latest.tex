\documentclass[10pt]{article}
\usepackage[utf8]{inputenc}
\usepackage[shortlabels]{enumitem}
\usepackage[margin=1in]{geometry}
\usepackage{setspace}
\usepackage{hyperref}
\usepackage{graphicx}
\usepackage{natbib}
\usepackage{booktabs}
\usepackage{subcaption}
\onehalfspacing
\setlength{\parskip}{1em}
\setlength{\parindent}{0pt}
\bibliographystyle{ecta}
\usepackage[T1]{fontenc}
\usepackage{titling}
\usepackage{amsmath}
\setlength{\droptitle}{-7em}
\addtolength\abovedisplayskip{-3in}
\addtolength\belowdisplayskip{-3in}

\title{City Location Choice and Household Productivity}
\preauthor{}
\postauthor{}
\author{Tan Sein Jone}
\predate{}
\postdate{}
\date{}
%\Today

\begin{document}
\doublespacing
\maketitle

\section{Introduction}

Cities capture a disproportionate amount of household location choices in the world. In 2018, 55\% of the world's population lived in urban areas, and this is expected to increase to 68\% by 2050. This trend is driven by the benefits cities provide, such as access to jobs, amenities, and social networks. While these advantages are often considered universal to all households, regardless of their occupation-specific skills, the benefits of each city clearly vary for different households. For instance, an auto worker from Detroit will not view New York as an equivalent substitute for a city to live in compared to a finance worker from New York. Given the billions of dollars spent on place-based policies, it is crucial to understand this dimension of comparative advantage, where a city can provide higher levels of productivity for households working in specific occupations.

A city's comparative advantage exists in three dimensions. First, there is a city-specific advantage that makes all occupations more productive in that city, primarily due to its size and population, which results in better amenities, more jobs, and a larger social network. Second, there is an occupation-specific advantage that is universal across all cities; some jobs are simply more productive than others, and cities specializing in these jobs will be observed as having higher overall choice shares of households. Third, there is a city-occupation specific advantage, where a city may have specific policies that make it more attractive for a particular occupation, such as tax breaks for mechanics in Chicago. It is crucial to distinguish between occupations and sectors: sectors are broad categories of jobs (e.g., manufacturing, services, agriculture), while occupations are specific jobs (e.g., mechanic, clerk). Cities can specialize in occupations by focusing on sectors that intensively use those occupations.

In this paper, we propose a model of city location choices based on a class of models in the quantitative trade literature first proposed by \cite{ek}. These models treat productivity as random draws from Fr\'{e}chet distributions, allowing for a non zero production of goods which will help us to account for small cities. \cite{redding} applies this model to an urban setting by loading city heterogeneity onto amenities, capturing the first dimension of comparative advantage. However, both these models rely on the independence assumption which treats that goods and cities are perfect substitutes for each other. \cite{lindandramondo} break this assumption by introducing a nested CES utility structure which allows for correlated draws of productivity within nests across countries. We apply this structure to the urban setting by treating occupations as independent nests, capturing substitution patterns where cities with similar occupational structures are treated as better substitutes for each other. This approach enables us to capture all three dimensions of comparative advantage.

We begin in the first section by adapting the model of \cite{lindandramondo} to the urban setting. Unlike their original paper where nests are implied, we explicitly model the nests as occupations. This specification yields several implications. First, we will be able to observe both overall city shares and city shares specific to each occupation, as well as city occupation compositions, which can be used as a measure of similarity between cities. Second, using this data, we will derive elasticities for all three dimensions of comparative advantage: city-specific, occupation-specific, and city-occupation specific elasticities. The last elasticity will serve as a sanity check for the direction of substitution, while city and occupation elasticities will be used to verify our claim of unbalanced substitution patterns that favor similar cities. In other words, shocks to cities with a certain occupational composition will have larger effects on cities with similar compositions due to correlations in productivity draws.

In the second section, we present our estimation strategy, exploiting variation over time to recover two important parameters of our model: the correlation parameter $\rho$ and the shape parameter $\theta$. As we will discuss later, these parameters determine the within and across occupation migration patterns. In this process, we will also recover our city, occupation, and city-occupation specific shifters.

In the third section, we apply our estimation equations to data from the American Community Survey at the metropolitan area and occupational level. We begin by presenting stylized facts about cities and their compositions, comparing similarities among top cities. We then discuss our estimation results, focusing on the correlation parameter $\rho$ and the shape parameter $\theta$, as well as our city, occupation, and city-occupation specific shifters. As we will demonstrate, $\rho$ can be interpreted as the degree of within-occupation migration, while $\rho$ represents the degree of across-occupation migration. Using this fully calibrated model, we conduct counterfactual analyses and simulate the effects of shocks at both the city and occupation level. This analysis allows us to evaluate the effectiveness of place-based policies and provide recommendations for designing such policies to maximize their impact.

\section{Model}

\begin{align*}
    Y_{ck} = Z_{ck}
\end{align*}

Consider a closed economy consisting of $N$ cities and $K$ occupations. Each city $c$ employs households in occupation $k$ to produce output $Y_{ck}$. We assume no trade costs between cities and that the price of the good produced is freely traded and priced under perfect competition. The price index of city $c$ is given by $\Phi_c$ and the wage of a household living in city $c$ working in occupation $k$ is given by $w_{ck} = \Phi_c Z_{ck}$ where $Z_{ck}$ is the productivity of the household. We assume that when a household chooses to work in a particular occupation, they will be randomly allocated to a sector within a given city. A household has symmetric CES preferences over cities and occupations and will choose a city-occupation pair that maximizes their utility. We will assume that households are perfectly mobile across cities and that there are no fixed costs to moving. The productivity of a household within a given city-occupation is drawn from a Fr\'{e}chet distribution. In \cite{ek}, the Fr\'{e}chet distribution is used to model productivity draws in a trade setting and has two main components. The shape parameter $\theta$ which determines the dispersion of draws across the distribution and reflects the heterogeneity of cities. $\theta$ will also determine the gains from city choices, ie how much households will gain from having more choices of cities to live in, which is analogous to gains from trade. The scale parameter $T_{ck}$ is a level shifter that determines the absolute advantage of a city-occupation pair, reflecting its attractiveness. Our specification so far assumes independent draws of productivity across cities, the joint distribution of which is given by:

\begin{align*}
    P[Z_1 < z_1, \dots, Z_N < z_N] & = \prod_{c}^{N} P[Z_c < z_c]                                 \\
                                   & = \exp \left\{ - \sum_{c}^{N} T_{ck} Z_c^{- \theta} \right\}
\end{align*}

This result implies that the choice shares of households for a given city is equal to the probability of households choosing that city. So far, we have stuck with the the base EK specification with the only modification being the mapping of productivity onto wages rather than prices which allows us to obtain city choice shares. We will now abandon the independence assumption and introduce a structure that allows for correlated draws across cities.

\begin{align*}
    P[Z_1 < z_1, \dots, Z_N < z_N] = \exp \left\{ - \left( \sum_{c}^{N} (T_{ck} Z_c^{- \theta})^{\frac{1}{1 - \rho}} \right)^{1 - \rho} \right\}
\end{align*}

Notice with this structure that the shape of the distribution is no longer purely determined by $\theta$ but also by the correlation parameter $\rho$. Where the distribution of productivity within a given occupation decreases as $\rho$ increases. Meaning that within a given occupation, the higher correlation will result in cities being more similar with each other in terms of productivity. When $\rho = 0$, the distribution reduces down to the original joint distribution with the shape being purely determined by $\theta$ and where draws are independent. As $\rho \rightarrow 1$, the draws become perfectly correlated and the productivity is identical across all cities.

\subsection{Cross Nested CES}

\begin{align}
    P[Z_1(\nu) < z, \dots, Z_N(\nu) < z] = \exp \left\{ - \sum_{k}^{K} \left[ \sum_{c}^{N} (T_{ck} Z_c^{- \theta})^{\frac{1}{1 - \rho_k}} \right]^{1 - \rho_k} \right\}
\end{align}

We will assume that productivity is distributed max stable multivariate Fr\'{e}chet with $T_{ck}$ being the scale parameter for city $c$ and occupation $k$. $\theta > 0$ is the shape parameter that determines the dispersion of draws across the distribution and $\rho_k$ is the occupation specific correlation parameter. As in \cite{lindandramondo}, we adopt a cross nested CES structure with a correlational structure within independent occupational nests. This has the characteristic of distributions not purely determined by $\theta$ but also by the correlation parameter. This is because the correlation parameter determines the degree of correlation between draws of productivity across cities.

In order to identify the three dimensions of comparative advantage, we will be separating out the scale parameter into three components where $T_{ck} = T_c T_k t_{ck}$. $T_c$ is the city specific scale parameter which captures the attractiveness of a city, shifting the distribution of all productivity draws within that city upwards. $T_k$ is the occupation specific scale parameter which is common across all cities, this captures the effect that any occupation has across all cities. $t_{ck}$ is the city-occupation specific scale parameter which captures the effect of a city on a specific occupation. This will give us the final dimension of comparative advantage.

\begin{align}
    Z = \max_c \left\{ \frac{Z_c}{\Phi_c} \right\}
\end{align}

A household's schedule of productivity is characterized by a vector of draws from different Fr\'{e}chet distributions for each city occupation pair. The realized productivity a household has in city $c$ is the occupation that maximizes the productivity has in the city. Unlike sequential games where households might pick a city before picking an occupation, we assume that households pick both simultaneously. A way to contextualize this is to think of households as already having an ideal occupation in mind when picking a city, hence $Z_c = \max_k \left\{ Z_{ck} \right\}$. This productivity is then scaled by the price index of a city to reflect how high prices reduces the purchasing power of the household within that city, hence making it less attractive.

\subsection{Equilibrium}

\begin{align}
    G(Z_1^{- \theta}, \dots, Z_N^{- \theta}) = \sum_{k}^{K} T_k \left[ \sum_{c}^{N} (t_{ck} Z_c^{- \theta})^{\frac{1}{1 - \rho_k}} \right]^{1 - \rho_k}
    \label{nested_ces}
\end{align}

We first define $Z_c^{- \theta} = (\gamma w_c)^{- \theta} T_c$ and $\gamma = \Gamma (\frac{\theta + 1 - \rho_k}{\theta})^{\frac{1}{1 - \rho_k}}$. Given our definition of $Z_c^{- \theta}$, $T_c$ will be absorbed within $Z_c^{- \theta}$, giving us the correlation function $G(Z_1^{- \theta}, \dots, Z_N^{- \theta})$ as defined in equation \ref{nested_ces}.

\begin{align*}
    \pi_c = \frac{Z_c^{- \theta} G_c(Z_1^{- \theta}, \dots, Z_N^{- \theta})}{G(Z_1^{- \theta}, \dots, Z_N^{- \theta})}
\end{align*}

Where $G_c (Z_1^{- \theta}, \dots, Z_N^{- \theta}) = \partial G(Z_1^{- \theta}, \dots, Z_N^{- \theta}) / \partial Z_c^{- \theta}$. The expression above gives us the city specific choice shares of households across all occupations. In order to evaluate the expression, we make the simplifying assumption that the correlation parameter is the same across all occupations $\rho_k = \rho$.

\begin{align*}
    \lambda_{k} = \sum_{c}^{N} \left( t_{ck} Z_{c}^{-\theta} \right)^{\frac{1}{1-\rho}}
\end{align*}

We introduce the above expression which measures the appeal of occupation $k$. Given that the expression does not include $T_k$, it does not capture the attractiveness of occupation $k$. Instead the expression captures the extent to which a given occupation $k$ exhibits city-specific productivities which are correlated with aggregate city-level attractiveness. The following definitions will be useful in simplifying the expression for city shares.

\begin{align*}
    \pi_{ck} = \frac{Z_c^{- \theta} G_c^k(Z_1^{- \theta}, \dots, Z_N^{- \theta})}{G^k(Z_1^{- \theta}, \dots, Z_N^{- \theta})}
\end{align*}

To obtain occupation specific choice shares, we can evaluate the derivative for city shares at the occupation level. More specifically, rather than $G(Z_1^{- \theta}, \dots, Z_N^{- \theta})$, we will be evaluating $G^k(Z_1^{- \theta}, \dots, Z_N^{- \theta}) = \sum_{c}^{N} (t_{ck} Z_c^{- \theta})^{\frac{1}{1 - \rho_k}}$. This will give us the following:

\begin{align}
    \pi_{ck} = \frac{(t_{ck} Z_c^{-\theta})^{\frac{1}{1 - \rho}}}{\lambda_k}
    \label{city_occuaption_shares}
\end{align}

Moving forward, it will be convenient for us to provide the following definitions. First, we define $\phi_{ck}$ as the comparative advantage of city $c$ in occupation $k$. This can be interpreted as an occupation's within city choice shares. As such $\phi_{ck}$ will be between 0 and 1. We define this formally as:

\begin{align}
    \phi_{ck} \equiv \frac{{t^{\frac{1}{1-\rho}}_{ck}}{T_{k}}\lambda_{k}^{-\rho}}{\sum\limits_{k}{t^{\frac{1}{1-\rho}}_{ck}}{T_{k}}\lambda_{k}^{-\rho}}
    \label{occupation_by_city_shares}
\end{align}

Second, we define an occupation specific choice share which takes into account both the size of each occupation, as measured by productivity $T_k$, and the extent to which individual city-occupation productivities are correlated with overall city attractiveness. This is defined as the following expression:

\begin{align}
    \omega_k \equiv \frac{{T_{k}}\lambda_{k}^{1-\rho}}{\sum\limits_{k}{T_{k}}\lambda_{k}^{1-\rho}}
    \label{occupation_shares}
\end{align}

From the perspective of discrete choice demand models, $\omega_k$ can be interpreted as the "market share" of occupation $k$. This is because it captures the extent to which the productivity of occupation $k$ is correlated with overall city attractiveness. The expression is normalized by the sum of all $\omega_k$ to ensure that the sum of all market shares is equal to 1.

\begin{align}
    \frac{\pi_{ck}}{\pi_c}\equiv \frac{\phi_{ck}}{\omega_k}
    \label{identity}
\end{align}

The expression above tells us the extent to which a given occupation $k$ is concentrated in city $c$ above and beyond the predicted concentration, given the size of the city $\pi_c$ is equivalent to the within-city choice share in occupation $k$ divided by the aggregate choice share of that occupation.

\begin{align*}
    \pi_{c} = \sum_{k}^{K} \frac{(Z_{c}^{-\theta} t_{ck})^{\frac{1}{1-\rho}}}{\lambda_k} \frac{T_k \lambda_{k}^{1 - \rho}}{\sum \limits_{k}^{K} T_k \lambda_{k}^{1-\rho}}
\end{align*}

\begin{align}
    \pi_c = \sum_{k}^{K} \pi_{ck} \omega_k
\end{align}

By using our definitions, we arrive at the above expression for city shares, which is now the sum of all occupation specific shares weighted by the market share of each occupation.

\subsection{City-Occupation Elasticities}

We now turn to the city-occupation derivatives: city $c$'s own technology elasticity and cross technology elasticity with respect to another city $c'$. This derivative will be approximated by assuming that each city is small, and therefore $\partial \ln \lambda_k / \partial \ln t_{ck} \approx 0$. That is, the aggregate "price index" associated with location choice in occupation $k$ is not affected by a marginal change in the productivity of a given city-occupation pair. We then obtain the following for the own technology shock:

\begin{align}
    \frac{\partial \ln \pi_{ck}}{\partial \ln t_{ck}} \approx \phi_{ck} \left( \frac{1}{1 - \rho} \right)
    \label{co_own_elasticity}
\end{align}

This elasticity is positive and increasing in a city's comparative advantage in occupation $k$. The intuition is that a city that is more attractive for occupation $k$ will see a larger increase in the choice share of that occupation in response to a marginal increase in the productivity of that occupation. The more specialized the city is in that occupation, the larger the effect of a productivity shock in that occupation. As $\rho \rightarrow 1$, the elasticity approaches infinity, as within any given occupation, if they become more correlated, cities within that given occupation become perfect substitutes for each other.

\begin{align}
    \frac{\partial\ln{\pi_{c}}}{\partial\ln{t_{{c'}k}}} = -{\pi_{c'}}{\phi_{ck}}\Big[1+\Big(\frac{\rho}{1-\rho}\Big)\Big(\frac{\pi_{ck}}{\pi_{c}}\Big)\Big]
    \label{co_cross_elasticity}
\end{align}

Turning our attention to the cross city technology elasticity, we see that the first term captures the CES element of consumer choice. As the city $c'$ increases in size and becomes increasingly competitive in that occupation, any shocks to this city-occupation pair will have an increasingly negative effects on all other cities.

How much the other cities will be affected is captured by the second term which captures the unbalanced substitution patterns. If a city is relatively concentrated in occupation $k$, then the affects of a shock in that given occupation will be felt more severely, this is shown by the ratio $\pi_{ck} / \pi_c$. Notice also that if $\rho = 0$, this elasticity simplified to the standard CES elasticity. As $\rho \rightarrow 1$, the elasticity approaches infinity and the correlated choices start to dominate.

\subsection{Aggregate Elasticities}

We now consider the elasticity of city shares with respect to aggregate city technology shocks $T_c$ and aggregate occupation shocks $T_k$. Beginning with the former, if we assume again that each cit is small and that $\partial \ln \lambda_k / \partial \ln T_c \approx 0$, we get the following for a city's own technology shock:

\begin{align*}
    \frac{\partial \ln \pi_c}{\partial \ln T_c} \approx \frac{1}{1 - \rho}
\end{align*}

This is simply the summation of city-occupation elasticities which was derived in expression \ref{co_own_elasticity}. And given that by definition $\sum \limits_k^K \phi_{ck} = 1$, we get the following derivations:

\begin{align*}
    \frac{\partial{\ln{\pi_{c}}}}{\partial\ln{T_{c}}} & = \sum\limits_{k}\frac{\partial{\ln{\pi_{c}}}}{\partial\ln{t_{ck}}} \\ &= \sum\limits_{k}\phi_{ck}\Bigg(\frac{1}{1-\rho}\Bigg) \\ &= \Bigg(\frac{1}{1-\rho}\Bigg)\sum\limits_{k}\phi_{ck} = \frac{1}{1-\rho}
\end{align*}

The result tells us that if $\rho = 0$ and there's no correlation between occupations, then the elasticity is simply 1. We get a similar story when deriving the cross city technology elasticity. Using the same assumption, we arrive at the result:

\begin{align}
    \frac{\partial\ln{\pi_{c}}}{\partial\ln{T_{c'}}} = -\pi_{c'}\Bigg[\sum\limits_{k}\phi_{{c'}k}\Big[1+\Big(\frac{\rho}{1-\rho}\Big)\Big(\frac{\pi_{ck}}{\pi_{c}}\Big)\Big]\Bigg]
\end{align}

Similarly, this is simply the sum over all city-occupation elasticities derived in expression \ref{co_cross_elasticity}. This is shown in the following derivation:

\begin{align*}
    \frac{\partial\ln{\pi_{c}}}{\partial\ln{T_{c'}}} & = \sum\limits_{k}\frac{\partial\ln{\pi_{c}}}{\partial\ln{t_{{c'}k}}} \\ &= \sum\limits_{k}\Bigg[-{\pi_{c'}}{\phi_{ck}}\Big[1+\Big(\frac{\rho}{1-\rho}\Big)\Big(\frac{\pi_{ck}}{\pi_{c}}\Big)\Big]\Bigg]\\ &= -\pi_{c'}\Bigg[\sum\limits_{k}\phi_{{c'}k}\Big[1+\Big(\frac{\rho}{1-\rho}\Big)\Big(\frac{\pi_{ck}}{\pi_{c}}\Big)\Big]\Bigg]
\end{align*}

The interpretation of this elasticity is the same as expression \ref{co_cross_elasticity}. The first term captures the CES element of consumer choice and the second term captures the unbalanced substitution patterns, with $\rho$ governing the strength of this unbalanced substitution. Now we simply sum over all occupations to get the aggregate elasticities.

Lastly, we derive the elasticity of city shares with respect to aggregate occupation shocks $T_k$. This is fairly straightforward as wel can simply evaluate the following:

\begin{align*}
    \frac{\partial\ln{\pi_{c}}}{\partial\ln{T_{k}}} & = \Big(\frac{\partial{\phi_{ck}}}{\partial{T_{k}}}\Big)\Big(\frac{T_{k}}{\phi_{ck}}\Big) - \Big(\frac{\partial{\omega_{k}}}{\partial{T_{k}}}\Big)\Big(\frac{T_{k}}{\omega_{k}}\Big) \\ &= \phi_{ck}-\omega_{k}
\end{align*}

By using the identity in expression \ref{identity}, we can rewrite the elasticity as the following:

\begin{equation}
    \frac{\partial\ln{\pi_{c}}}{\partial\ln{T_{k}}} = {\omega_{k}}\Bigg[\frac{\pi_{ck}}{\pi_{c}}-1\Bigg]
\end{equation}

You should first notice that the second term in brackets can be either positive or negative depending on the city's relative concentration in that occupation. If the city is relatively concentrated and $\pi_{ck} / \pi_c > 1$, then the city will face a negative shock. If the city is relatively less concentrated and $\pi_{ck} / \pi_c < 1$, then the city will face a negative shock. We can think of this as certain occupations start to become more attractive, cities that are less concentrated in that occupation will face a negative shock as households start to move to cities that are more concentrated in that occupation. This second term is scaled by the market share of that occupation, $\omega_k$, which is simply the market share of that occupation across all cities.

\section{Estimation Strategy}

% \subsection{Shifters}

% Given our model, we have 4 sets of shares that we can consider. City shares $\pi_c$, city shares for each occupation $\pi_{ck}$, occupation market shares $\omega_k$ and occupation shares for each city $\phi_{ck}$. In order for us to recover city and city-occupation specific shifters, consider the following estimation equation derived from expression \ref{city_occuaption_shares}:

% \begin{align*}
%     \ln \pi_{ck} & = \frac{1}{1 - \rho} \ln t_{ck} - \frac{1}{1 - \rho} \ln Z_c^{- \theta} - \ln \lambda_k \\
%                  & = \hat{\epsilon_{ck}} + + \hat{\eta_c^1} + \hat{\delta_k^1}
% \end{align*}

% This will be a simple OLS regression with city and occupation level fixed effects on city shares by occupation. With $\hat{\eta_c^1}$ being our city specific shifter, $\hat{\delta_k^1}$ being our occupation specific shifter and $\hat{\epsilon_{ck}}$ being our residuals.

% \begin{align*}
%     t_{ck}         & = exp \left(  (1 - \rho) \hat{\epsilon_{ck}}\right)  \\
%     \lambda_k      & = exp \left( (1 - \rho) \hat{\eta_c^1} \right)       \\
%     Z_c^{- \theta} & = exp \left( - \hat{\delta_k^2 }\right)              \\
%     T_c            & = Z_c^{- \theta} \left( \gamma \Phi_c \right)^\theta
% \end{align*}

% The equations above allow us to recover our city and city-occupation specific scale parameters. $\lambda_k$ will be used shortly to recover our occupation specific scale parameters.

% \begin{align*}
%     \ln \phi_{ck} & = \frac{1}{1 - \rho} \ln t_{ck} + \ln T_k \lambda_k^{- \rho} - \ln \left( \sum_k^K t_{ck}^{\frac{1}{1 - \rho}} T_k \lambda_k^{- \rho} \right) \\
%                   & = \hat{\epsilon_{ck}} + \hat{\delta_k^2} + \hat{\eta_c^2}
% \end{align*}

% Similarly, we will be using our occupation shares defined in expression \ref{occupation_by_city_shares} and have it be regressed against city and occupation fixed effects. Given that we have already identified $\lambda_k$, we can use this to recover our occupation specific scale parameters which is given by the equation below:

% \begin{align*}
%     T_k = \lambda_k^\rho e^{\hat{\delta_k^2}}
% \end{align*}

\subsection{Within-Occupation Migration}

Notice that for us to estimate our shifters, we have to first recover our $\rho$ parameter. To do so, we exploit a variation in city-occupation shares over time. Consider the following equation at the city-occupation-year level. From expression \ref{city_occuaption_shares}, we can get the following:

\begin{align*}
    \ln \pi_{ck} & = \frac{1}{1 - \rho} \ln t_{ckt} - \frac{1}{1 - \rho} \ln Z_{ct}^{- \theta} - \ln \lambda_{kt}
\end{align*}

Now we assume the following relationship productivity and wages at the city-occupation level: $t_{ckt} = w_{ckt} \eta_{ckt}$. Where $w_{ckt}$ are pages paid and $\eta_{ckt}$ is productivity. This will give us the following model:

\begin{align*}
    \ln \pi_{ckt} = \left( \frac{1}{1 - \rho} \right) \ln w_{ckt} + \delta_{ct} + \delta_{kt} + \eta_{ckt}
\end{align*}

With some abuse of notation, we have defined $\eta_{ckt} = \ln \eta_{ckt}$. We now take differences over time to eliminate level shifts.

\begin{align*}
    \Delta_t \ln \pi_{ckt} = \left( \frac{1}{1 - \rho} \right) \Delta_t \ln w_{ckt} + \delta_{ct} + \delta_{kt} + \Delta_t \eta_{ckt}
\end{align*}

We can estimate this equation by regressing wages on city occupation shares and using city and occupation fixed effects at the city-occupation level for every year with $\delta_{ct}$ being city fixed effects and $\delta_{kt}$ being occupation fixed effects. Given that $\hat{\rho} \in (0, 1)$, we will end up with the restriction $\hat{\beta} \in [1, +\infty)$.

With this regression, we are testing whether migration decisions are primarily within occupations rather than across occupations. If the correlation parameter is close to 1, then migration decisions are purely within occupations. Notice that as $\hat{\rho}$ increases, $\hat{\beta}$ increases. From this perspective, we can interpret $\rho$ as governing within-occupation migration and $\theta$ as governing across-occupation migration.

\begin{align*}
    Y_{st} = A_{st} \prod_{k}^{K} Q_{kst}^{\gamma_{ks}}
\end{align*}

One concern with the initial specification is the endogeneity of wages with the error term,more explicitly $cov[w_{ckt}, \eta_{ckt}] \neq 0$. In order to instrument for the change in wages, we will use a shift share in which we make use of sectors. Here, we define a Cobb-Douglas strcture of production where the intensity of occupation $k$ in sector $s$ is given by $\gamma_{ks}$ such that $\sum_{k}^{K} \gamma_{ks} = 1$. We will use the shift share of occupation $k$ in sector $s$ as an instrument for the change in wages. $A_{st}$ will be the total factor productivity of sector $s$ in year $t$.

\begin{align*}
    w_{ckt}^{IV} = \sum_{s}^{S} \gamma_{ks} \kappa_{cs} A_{st}
\end{align*}

where $\kappa_{cs}$ is the  share of occupation $k$ in sector $s$ in city $c$. This will give us the following first stage regression:

\begin{align*}
    \ln w_{ckt}^{IV} = \beta^{IV} \ln w_{ckt}^{IV} + \delta_{ct} + \delta_{kt}^{IV} + \eta_{ckt}^{IV}
\end{align*}

We will then use the predicted values of $\ln w_{ckt}^{IV}$ as our instrument in the second stage regression.

\begin{align*}
    \ln \pi_{ckt} = \left( \frac{1}{1 - \rho} \right) \ln w_{ckt}^{IV} + \delta_{ct} + \delta_{kt} + \eta_{ckt}
\end{align*}

\subsection{Across Occupation Migration}

Given that we've identified $\rho$, we can now proceed to estimate our shape parameter $\theta$ and in the process recover all our scale parameters. We start with occupation market shares $\omega_k$ in expression \ref{occupation_shares}. By taking relative shares, we get the following:

\begin{align*}
    \frac{\omega_{kt}}{\omega_{k't}} = \frac{T_{kt} \lambda_{kt}^{1 - \rho}}{T_{k't} \lambda_{k't}^{1 - \rho}}
\end{align*}

Subject to normalization to some base occupation year, we can estimate $T_{kt} \lambda_{kt}^{1 - \rho} \forall (k, t)$. We can then use this to recover the denominator in $\omega_k$ and get $\sum_{k}^{K} T_{kt} \lambda_{kt}^{1 - \rho} \forall t$. This will be useful in recovering our city and city occupation specific scale parameters. Now consider the city-occupation shares from expression \ref{city_occuaption_shares} taken in logs and transformed into a form with fixed effects:

\begin{align*}
    \ln \pi_{ckt} & = \frac{1}{1 - \rho} \ln t_{ckt} - \frac{1}{1 - \rho} \ln Z_{ct}^{- \theta} - \ln \lambda_{kt} \\
                  & = \hat{\epsilon}_{ckt} + \hat{\delta}_{ct} + \hat{\delta}_{kt}
\end{align*}

such that,

\begin{align*}
    \hat{\lambda}_{kt} & = \exp \left( - \hat{\delta}_{kt} \right)             \\
    \hat{t}_{ckt}      & = \exp \left( (1 - \rho) \hat{\epsilon}_{ckt} \right) \\
\end{align*}

We can then use the estimated $\hat{\lambda}_{kt}$ to recover our occupation specific scale parameters:

\begin{align*}
    \hat{T}_{kt} = \frac{\widehat{\sum_{k}^{K} T_{kt} \lambda_{kt}^{1 - \rho}}}{\hat{\lambda}_{kt}^{1 - \rho}}
\end{align*}

Now we need some way for us to recover productivity in order to estimate the shape parameter $\theta$. Consider the following modification to city shares using our recovered parameters:

\begin{align*}
    \tilde{\pi}_{ct} & = \pi_{ct} \frac{\sum_{k}^{K} T_{kt} \lambda_{kt}^{- \rho}}{\sum_{k}^{K} t_{ckt}^{\frac{1}{1 - \rho}} T_{kt} \lambda_{kt}^{1 - \rho}} \\
                     & = Z_{ct}^{- \frac{\theta}{1 - \rho}}
\end{align*}

We now take logs and difference the equation to eliminate the level shift. Recalling our definition of $Z_c^{- \theta} = (\gamma w_c)^{- \theta} T_c$, we get the following estimation equation:

\begin{align*}
    \Delta_t \ln \hat{\pi}_{ct} = \hat{\beta} \Delta_t \ln w_{ct} + \Delta_t \hat{\eta}_{ct}
\end{align*}

The differencing will eliminate $\gamma$, which is constant over time, leaving us with a regression of changes in wages and city fixed effects over time on changes in city shares. The coefficient $\hat{\beta}$ will be our estimate of $ - \theta / (1 - \rho)$, which when rearranged gives us $\hat{\theta} = - \hat{\beta} (1 - \hat{\rho})$. Note the relationship between $\theta$ and $\rho$ is such that as $\rho$ increases, $\theta$ decreases. This indicates that as the strength of within occupation migration increases, the across occupation migration becomes less important, reinforcing our interpretation of $\rho$ as governing within-occupation migration and $\theta$ as governing across-occupation migration.

\begin{align*}
    \hat{T}_{ct} = exp \left( (1 - \rho) \hat{\eta}_{ct} \right)
\end{align*}

Using this regression, we also recover city level shifters from fixed effects estimates.

\section{Empirical Results}

\subsection{Data}

We will be using data from the American Community Survey (ACS) which is a survey conducted by the US Census Bureau. The ACS is a survey that collects data on the metropolitan area and occupational level. We will be using data from 2010 to 2022. The ACS provides data on the household level, detailing wages, metropolitan area, occupation and sector of employment based on NAICS codes. This will be used to construct our share data $\pi_c$ $\pi_{ck}$, $\omega_k$, $\phi_{ck}$ and $\phi_{cs}$ as well as obtain wage data $w_{ckt}$. We will also be using data from the Bureau of Labour Statistics to obtain total factor productivity data at the sector level $A_{st}$.

% \begin{figure}[!htb]
%     \centering
%     \includegraphics[width=\textwidth]{../../estimations/graphs/city_employment_share.png}
%     \caption{City Employment Share}
%     \label{employment_city_share}
% \end{figure}

% \begin{figure}[!htb]
%     \centering
%     \includegraphics[width=\textwidth]{../../estimations/graphs/top_25_city_heatmap.png}
%     \caption{Occupation Shares by Cities}
%     \label{top_25_city_heatmap}
% \end{figure}

\begin{figure}[!htb]
    \centering
    Figure 1: Summary Statistics of Metropolitan Statistical Areas (MSA) in 2019\par\medskip
    \begin{minipage}{0.48\textwidth}
        \centering
        \includegraphics[width=\textwidth]{../../estimations/graphs/city_employment_share.png}
        \subcaption{Employment Share by MSA in 2019}
        \label{employment_city_share}
    \end{minipage}\hfill
    \begin{minipage}{0.48\textwidth}
        \centering
        \includegraphics[width=\textwidth]{../../estimations/graphs/top_25_city_heatmap.png}
        \subcaption{Composition of Top 10 MSAs in 2019}
        \label{top_25_city_heatmap}
    \end{minipage}
    \caption*{\small\textit{Note: We will be referring to MSAs as cities throughout the paper.}}
\end{figure}

From the data in Figure \ref{employment_city_share}, we see that a few metropolitan areas own a disproportionate share of employment which illustrates the size effects of cities to attract employment and households. Intuitively, we should expect these cities to be very different from each other in terms of their specialization, otherwise we would expect the largest city with a particular composition to dominate in the share of households. When zooming into the compositions of the top 10 cities however, we see a different story. In Figure \ref{top_25_city_heatmap}, we see that the top 25 cities are very similar with each other in terms of their specialization. With all cities specializing in occupations like management and professional services. While there is some heterogeneity in the composition of cities, the specialization of cities is not as pronounced as we would expect. This indicates to us that the occupation shifter will play a large role in determining the attractiveness of a city.

\subsection{Results}

To recover our $\rho$ parameter, we run a two stage least squares regression with a shift share instrument. We find that this regression yields a high F-Statistic and a significantly higher estimate for $\hat{\beta}$ compared to ols\footnote{Refer to appendix 8.2 for full regression table}. This ultimately results in a fairly high estimate of $\hat{\rho} = 0.67$. The result means that draws across cities within any given occupation id highly correlated and indicates to us that people are very likely to migrate within occupations. This is consistent with the idea that people are more likely to move to cities where they can find jobs in their occupation.

When taking a look at the regression of wages against city shares, we $\hat{\beta}$ to be fairly low, giving us a resulting $\theta$ of 0.02. This is surprising for a couple of reasons. First, it indicates to us that the shape of the Fr\'{e}chet distributions are mainly controlled by the correlation parameter $\rho$, meaning that across labor migration, meaning migration that is purely due to cities is not that important. Second, it means that migration is mainly within occupations rather than across occupations. Resulting in a dynamic where people are actually very willing to migrate, but only to cities that have the occupations they are looking for.

\begin{table}[!htb]
    \centering
    \caption{Estimations for $T_c$ and $T_k$ for 2019}
    \begin{minipage}{0.4\textwidth}
        \centering
        \caption*{(A) Occupation Shifters}
        \begin{tabular}{lr}
\toprule
Occupation & Occupation Shifter \\
\midrule
Management & 0.849566 \\
Professional & 0.743607 \\
Trades & 0.610380 \\
Service & 0.344887 \\
Sales & 0.318461 \\
Cleaners & 0.090246 \\
Primary & 0.066206 \\
Security & 0.066183 \\
\bottomrule
\end{tabular}

    \end{minipage}
    \hfill
    \begin{minipage}{0.55\textwidth}
        \centering
        \caption*{(B) Top City Shifters}
        \begin{tabular}{lr}
\toprule
City & City Shifter \\
\midrule
New York-Newark-Jersey City & 3.029541 \\
Los Angeles-Long Beach-Anaheim & 2.645708 \\
Chicago-Naperville-Elgin & 2.333439 \\
Dallas-Fort Worth-Arlington & 2.158189 \\
Houston-The Woodlands-Sugar Land & 2.093633 \\
Philadelphia-Camden-Wilmington & 2.028458 \\
Washington-Arlington-Alexandria & 2.007037 \\
Atlanta-Sandy Springs-Roswell & 2.001475 \\
\bottomrule
\end{tabular}

    \end{minipage}
    \caption*{\small\textit{Note: Occupation and city shifters are not directly comparable. Recall that occupations are scaled only by $\rho$ while occupations are scaled by $\rho$ and $\theta$. Comparisons should only be relative and be made within each table.}}
\end{table}

In 2019, we find that occupation shifters are generally lower compared to city shifters. However, since cities are scaled by an additional factor of theta, the resulting effect from occupation shifters remains higher than city shifters. For occupation shifters, it's unsurprising that professional and management services have the highest values, given the proportion of households employed in these occupations. Conversely, primary sector occupations have the lowest shifters. Coupled with the fact that these occupations are generally performed in cities with low shifters themselves, this results in cities specializing in primary occupations having low shares of households. It's important to note, however, that while a city-occupation pair may be generally unattractive to most households, those who specialize in these occupations will be very willing to move to cities with occupations matching their abilities.

Most dense metropolitan areas tend to specialize in occupations with high shifters, explaining the large pattern of rural-urban migration to these cities in the past. However, now that most cities have developed similar occupational compositions, we see large cities competing with each other for household shares, which is why no particular city is universally most attractive to all households. While large cities tend to have similar overall compositions, they still manage to distinguish themselves by having distinct compositions in other occupations. For example, although New York may be concentrated in management and professional services, it has a higher concentration in sales compared to Los Angeles. This suggests that these cities are large because of their dominant occupations, but they don't compete purely based on scale with other cities.

Unsurprisingly, cities with the highest shifters are dense metropolitan areas such as New York and Los Angeles, reflecting the large populations they attract. Notable in our estimates is the variance in shifter values between cities, with some shifters being up to three times larger than others, indicating significant heterogeneity in city attractiveness. This makes sense as large cities tend to offer more and better amenities as well as better access to jobs, making them more attractive to households. We also find that city shifters tend to be fairly stable over time, with changes generally below 10\% over the 10-year period. Given the time required for cities to build infrastructure and other amenities, in addition to needing to differentiate themselves from other cities, this stability is not surprising.

\section{Quantitative Exercises}

Using our parameter and shifter estimates, we can now perform two counterfactual exercises. First, we will simulate a city-specific shock to observe migration patterns across cities, focusing on the unbalanced substitution patterns that favor similar cities. Second, we will examine the migration patterns resulting from a shock to a particular occupation. This effectively simulates the effects of a trade shock that impacts some occupations more than others, such as the China shock. These exercises will provide insights into how different types of economic shocks affect urban migration and occupation choices.

\subsection{City Specific Shock}

\begin{figure}[!htb]
    \centering
    \caption{City Compositions Benefitting from Shocks}
    \begin{minipage}{0.48\textwidth}
        \centering
        \includegraphics[width=\textwidth]{../../estimations/graphs/ny_counter.png}
        \subcaption{Top 10 Cities That Benefitted Most from the New York Shock}
        \label{ny_change_graph}
    \end{minipage}\hfill
    \begin{minipage}{0.48\textwidth}
        \centering
        \includegraphics[width=\textwidth]{../../estimations/graphs/vp_counter.png}
        \subcaption{Top 10 Cities That Benefitted Most from the Visalia-Porterville Shock}
        \label{vp_change_graph}
    \end{minipage}
\end{figure}

Our results from the previous section suggest that while we may observe some migration to large cities purely due to their size, we should expect to see more migration to cities with similar occupational makeups. For this exercise, we induce a negative shock of 50\% to all occupations in New York, a city primarily focused on management and professional services occupations. Analyzing the results, we find that larger cities benefit most from this shock, which is not surprising. However, what's particularly interesting is that the cities benefiting most are those with similar occupational compositions, as shown in Figure \ref{ny_change_graph}. That said, when considering all cities with similar occupational compositions, the size of the city becomes the deciding factor in where people move. Given that all top cities are very similar, it's unsurprising that the vast majority of people will move to the largest city.

Now, let's examine the effects of shocking a city that specializes in an occupation least likely to be found in large cities. For this case, we'll shock Visalia-Porterville, CA, a city highly specialized in primary occupations. Households specializing in these occupations are typically very willing to move to even relatively unattractive cities, provided they offer opportunities in their field. Our results confirm this expectation: we observe minimal movement to attractive but dissimilar cities, and substantial movement to cities that have a high concentration of primary occupations. This demonstrates the strong pull of occupation-specific opportunities, even when they exist in less conventionally desirable urban areas.

\subsection{Occupation Specific Shock}

\begin{figure}[!htb]
    \centering
    \caption{City Compositions of the Top 10 Cities Benefitting from the Management Occupation Shock}
    \includegraphics[width=\textwidth]{../../estimations/graphs/man_counter.png}
    \label{man_change_graph}
\end{figure}

What happens when there's a shock to particular occupations? We've established that households move to cities with similar occupational profiles when there's a negative shock to specific cities, but how do they respond when a shock affects all cities in a particular occupation? To explore this, we induce a negative shock of 50\% to management occupations across all cities.

Initially, we expected occupations with the largest shifter to be most attractive to households looking to switch careers. Surprisingly, we found an even redistribution of households across all occupations. Even more intriguing are the destination choices of these households. While we observe significant movement to large cities, we also see substantial migration to cities with different occupational compositions, even if they're relatively small. This aligns with the notion that households are willing to relocate and change occupations to secure employment, even at the cost of overall productivity.

This outcome challenges our initial hypothesis that trade shocks would result in minimal migration due to limited opportunities elsewhere. However, we believe these results can be rationalized in two ways:

\begin{itemize}
    \item The model lacks an unemployment option, potentially failing to capture households' decisions to remain jobless in certain cities while hoping for future employment.
    \item The model doesn't account for frictions like moving costs and retraining expenses. While our results may represent the eventual equilibrium, the model can't capture the potentially lengthy transitionary period.
\end{itemize}

These findings highlight the complex relationship between occupation-specific shocks, city attractiveness, and household migration decisions. They also point to areas where our model could be refined in future research to better capture real-world dynamics.

\section{Discussion}

Given the empirical and counterfactual results, we believe there are a couple of important policy implications to consider. Place-based policies that aim to make a city overall more productive in all occupations will generally be ineffective in attracting migration. This is because households tend to pick cities based on their occupations rather than the city itself. A more effective place-based policy should target either occupations that the city already specializes in, or occupations that have a high productivity shifter. Importantly, to increase a city's relative competitiveness, policymakers should focus on occupations that make it less similar to other cities in order to decrease direct competition.

Policies aimed at responding to trade shocks should be focused on transitioning households to other occupations. Unlike city shocks where households can simply move to other cities for better opportunities, trade shocks will result in households being persistently unemployed due to the lack of opportunities in their given occupation specialization. To tackle this form of unemployment, governments need to facilitate the transition of these households to other occupations. This can be done through retraining programs or by providing incentives for firms to hire these workers. Such measures will not only help households find jobs, but will also help cities that are highly specialized in these affected occupations to attract more households. By adopting these targeted approaches, policymakers can more effectively address the challenges posed by both localized economic changes and broader trade shocks, ultimately fostering more resilient urban economies and labor markets.

\section{Conclusion}

In this paper, we have developed a model of city and occupation choice that enables us to recover city and occupation-specific shifters and substitution parameters. Our findings demonstrate that occupations play a more significant role in migration decisions compared to cities themselves, with households more likely to relocate to cities with similar occupational compositions. We have also shown that within-occupation migration is more prevalent than across-occupation migration, consistent with the notion that people are more inclined to move to cities where they can find jobs in their current occupation. In the future, we hope to extend this model to include a housing market and unemployment which will allow up to explain housing prices in relatively low productivity cities as well as persistent unemployment in certain cities.

\newpage
\bibliography{latest}

\newpage

\section{Appendix}

\subsection{Derivations}

\noindent\textbf{City Choice Shares $\pi_{c}$}: As discussed in Lind and Ramondo (2023), the choice shares can be expressed as:

\begin{equation*}
    \pi_{c}=\frac{Z_{c}^{-\theta}{G_{c}}}{G()}
\end{equation*}

We define $G()$ as the following, given the definition of $T_{ck}$ and $\lambda_{k}$ provided in the main text:

\begin{equation*}
    G({Z_{1}}^{-\theta},...,{Z_{N}}^{-\theta})=\sum\limits_{k}\Big[\sum\limits_{c}^{N}({T_{ck}}{Z_{c}^{-\theta}})^{\frac{1}{1-\rho_{k}}}\Big]^{1-\rho_{k}} = \sum\limits_{k}{T_{k}}\lambda^{1-\rho}_{k}
\end{equation*}

$G_{c}()$ is simply the derivative of $G()$ with respect to $Z_{c}^{-\theta}$.

\begin{align*}
    \frac{\partial{G()}}{\partial{Z_{c}^{-\theta}}} & = \sum\limits_{k}\frac{\partial}{\partial{Z_{c}^{-\theta}}}\Big[\sum\limits_{c}^{N}({T_{ck}}{Z_{c}^{-\theta}})^{\frac{1}{1-\rho}}\Big]^{1-\rho} \\ &= \sum\limits_{k}\frac{\partial}{\partial{Z_{c}^{-\theta}}}\Big[{T^{\frac{1}{1-\rho}}_{k}}\sum\limits_{c}^{N}({t_{ck}}{Z_{c}^{-\theta}})^{\frac{1}{1-\rho}}\Big]^{1-\rho} \\ &= \sum\limits_{k}{T_{k}}\Big[(1-\rho)\lambda^{-\rho}_{k}\frac{\partial{\lambda_{k}}}{\partial{Z_{c}^{-\theta}}}\Big] \\ &= \sum\limits_{k}{T_{k}}\Big[(1-\rho)\lambda^{-\rho}_{k}(\frac{1}{1-\rho}){t^{\frac{1}{1-\rho}}_{ck}}(Z_{c}^{-\theta})^{\frac{\rho}{1-\rho}}\Big]\\ &= (Z_{c}^{-\theta})^{\frac{\rho}{1-\rho}}\sum\limits_{k}{T_{k}}{t^{\frac{1}{1-\rho}}_{ck}}\lambda_{k}^{-\rho}
\end{align*}

Putting these three terms together, we derive the choice shares:

\begin{align*}
    \pi_{c} & = \frac{Z_{c}^{-\theta}{G_{c}}}{G()} \\ &= (Z_{c}^{-\theta})^{\frac{1}{1-\rho}}\Bigg[\frac{\sum\limits_{k}{t^{\frac{1}{1-\rho}}_{ck}}{T_{k}}\lambda_{k}^{-\rho}}{\sum\limits_{k}{T_{k}}\lambda_{k}^{1-\rho}}\Bigg]
\end{align*}

\noindent\textbf{Own Occupation-Specific Elasticity:} We wish to evaluate $\partial\ln{\pi_{c}}/\partial\ln{t_{ck}}$, under the assumption that $\partial\ln{\lambda_{k}}/\partial\ln{t_{ck}}=0$. Notice this assumption yields the following:

\begin{equation*}
    \frac{\partial\ln{\pi_{c}}}{\partial\ln{t_{ck}}} \approx \frac{\partial\ln{\pi_{ck}}}{\partial\ln{t_{ck}}} - \frac{\partial\ln{\phi_{ck}}}{\partial\ln{t_{ck}}}
\end{equation*}

given the definitions of $\pi_{ck}$ and $\phi_{ck}$ provided in the text.

Notice that this only leaves one term to evaluate if we take the logarithm of $\pi_{c}$ and the derivative of this logarithm with respect to $\ln{t_{ck}}$:

\begin{align*}
    \frac{\partial\ln{\pi_{c}}}{\partial\ln{t_{ck}}} & \approx \frac{\partial\ln[{\sum\limits_{k}{t^{\frac{1}{1-\rho}}_{ck}}{T_{k}}\lambda_{k}^{-\rho}}]}{\partial\ln{t_{ck}}} \\ &= \Bigg(\frac{\partial[{\sum\limits_{k}{t^{\frac{1}{1-\rho}}_{ck}}{T_{k}}\lambda_{k}^{-\rho}}]}{\partial{t_{ck}}}\Bigg)\Bigg(\frac{t_{ck}}{{\sum\limits_{k}{t^{\frac{1}{1-\rho}}_{ck}}{T_{k}}\lambda_{k}^{-\rho}}}\Bigg)\\ &= \Bigg(\frac{1}{1-\rho}\Bigg)\Bigg(\frac{t_{ck}^{\frac{1}{1-\rho}}{T_{k}}\lambda_{k}^{-\rho}}{t_{ck}}\Bigg)\Bigg(\frac{t_{ck}}{{\sum\limits_{k}{t^{\frac{1}{1-\rho}}_{ck}}{T_{k}}\lambda_{k}^{-\rho}}}\Bigg) \\ &= \Big(\frac{1}{1-\rho}\Big)\Bigg[\frac{{t^{\frac{1}{1-\rho}}_{ck}}{T_{k}}\lambda_{k}^{-\rho}}{\sum\limits_{k}{t^{\frac{1}{1-\rho}}_{ck}}{T_{k}}\lambda_{k}^{-\rho}}\Bigg]\\ &= \frac{\phi_{ck}}{1-\rho}
\end{align*}

\noindent\textbf{Cross-City Occupation-Specific Elasticity:} We wish to evaluate $\partial\ln{\pi_{c}}/\partial\ln{t_{{c'}k}}$. Notice that this elasticity can be expressed as the following two terms:

\begin{equation*}
    \frac{\partial\ln{\pi_{c}}}{\partial\ln{t_{{c'}k}}} = {t^{\frac{1}{1-\rho}}_{ck}}{T_{k}}\Big(\frac{\partial\lambda_{k}^{-\rho}}{\partial{t_{{c'}k}}}\Big)\Big(\frac{t_{{c'}k}}{{\sum\limits_{k}{t^{\frac{1}{1-\rho}}_{ck}}{T_{k}}\lambda_{k}^{-\rho}}}\Big) - {T_{k}}\Big(\frac{\partial\lambda_{k}^{1-\rho}}{\partial{t_{{c'}k}}}\Big)\Big(\frac{t_{{c'}k}}{{\sum\limits_{k}{T_{k}}\lambda_{k}^{1-\rho}}}\Big)
\end{equation*}

It will be useful to first define the derivative of $\lambda_{k}$ with respect to our variable of interest, $t_{{c'}k}$:

\begin{equation*}
    \frac{\partial{\lambda_{k}}}{\partial{t_{{c'}k}}} = \Big(\frac{1}{1-\rho}\Big)\Big(\frac{1}{t_{{c'}k}}\Big)[{t_{{c'}k}}(Z_{c'}^{-\theta})]^{\frac{1}{1-\rho}}
\end{equation*}

We can now use chain rule in order to evaluate our elasticity of interest. The first term becomes the following:

\begin{align*}
    {t^{\frac{1}{1-\rho}}_{ck}}{T_{k}}\Big(\frac{\partial\lambda_{k}^{-\rho}}{\partial{t_{{c'}k}}}\Big)\Big(\frac{t_{{c'}k}}{{\sum\limits_{k}{t^{\frac{1}{1-\rho}}_{ck}}{T_{k}}\lambda_{k}^{-\rho}}}\Big) & = {t^{\frac{1}{1-\rho}}_{ck}}{T_{k}}(-\rho)(\lambda_{k}^{-\rho-1})\Big(\frac{1}{1-\rho}\Big)\Bigg(\frac{[{t_{{c'}k}}(Z_{c'}^{-\theta})]^{\frac{1}{1-\rho}}}{{\sum\limits_{k}{t^{\frac{1}{1-\rho}}_{ck}}{T_{k}}\lambda_{k}^{-\rho}}}\Bigg) \\ &= -\Bigg(\frac{\rho}{1-\rho}\Bigg)\Bigg(\frac{[{t_{{c'}k}}(Z_{c'}^{-\theta})]^{\frac{1}{1-\rho}}}{\lambda_{k}}\Bigg)\Bigg(\frac{t^{\frac{1}{1-\rho}}_{ck}{T_{k}}{\lambda^{-\rho}_{k}}}{{\sum\limits_{k}{t^{\frac{1}{1-\rho}}_{ck}}{T_{k}}\lambda_{k}^{-\rho}}}\Bigg)\\ &= -\Bigg(\frac{\rho}{1-\rho}\Bigg)\Bigg(\frac{[{t_{{c'}k}}(Z_{c'}^{-\theta})]^{\frac{1}{1-\rho}}}{\lambda_{k}}\Bigg){\phi_{ck}}\\ &= -\Big(\frac{\rho}{1-\rho}\Big){\pi_{{c'}k}}{\phi_{ck}}
\end{align*}

The second term can be derived in the following way:

\begin{align*}
    - {T_{k}}\Big(\frac{\partial\lambda_{k}^{1-\rho}}{\partial{t_{{c'}k}}}\Big)\Big(\frac{t_{{c'}k}}{{\sum\limits_{k}{T_{k}}\lambda_{k}^{1-\rho}}}\Big) & = - {T_{k}}(\lambda_{k}^{-\rho})[{t_{{c'}k}}(Z_{c'}^{-\theta})]^{\frac{1}{1-\rho}}\Big(\frac{1}{{\sum\limits_{k}{T_{k}}\lambda_{k}^{1-\rho}}}\Big) \\ &= - \Bigg(\frac{[{t_{{c'}k}}(Z_{c'}^{-\theta})]^{\frac{1}{1-\rho}}}{\lambda_{k}}\Bigg)\Big(\frac{{T_{k}}{\lambda_{k}^{1-\rho}}}{{\sum\limits_{k} T_k \lambda_{k}^{1-\rho}}}\Big)\\ &= -\pi_{{c'}k}{\omega_{k}}
\end{align*}

Putting these together, we derive the cross-city occupation-specific elasticity as the following:

\begin{equation*}
    \frac{\partial\ln{\pi_{c}}}{\partial\ln{t_{{c'}k}}} = -{\pi_{{c'}k}}\Big[\omega_{k}+\Big(\frac{\rho}{1-\rho}\Big)\phi_{ck}]
\end{equation*}

Notice that this is equivalent to the following, given the identity linking $\pi_{c}$, $\pi_{ck}$, $\phi_{ck}$, and $\omega_{k}$.

\begin{equation}
    \frac{\partial\ln{\pi_{c}}}{\partial\ln{t_{{c'}k}}} = -{\pi_{c'}}{\phi_{ck}}\Big[1+\Big(\frac{\rho}{1-\rho}\Big)\Big(\frac{\pi_{ck}}{\pi_{c}}\Big)\Big]
\end{equation}

\noindent\textbf{Cross-City Aggregate Elasticity:} We wish to evaluate $\partial\ln{\pi_{c}}/\partial\ln{T_{c'}}$. Notice that:

\begin{equation*}
    \frac{\partial\ln{\pi_{c}}}{\partial\ln{T_{c'}}} = \Bigg(\frac{\sum\limits_{k}{{t^{\frac{1}{1-\rho}}_{ck}}}{T_{k}}{\lambda^{-\rho}_{k}}}{\partial{T_{c'}}}\Bigg)\Bigg(\frac{T_{c'}}{\sum\limits_{k}t^{\frac{1}{1-\rho}}_{ck}{T_{k}}{\lambda^{-\rho}_{k}}}\Bigg) - \Bigg(\frac{\partial\sum\limits_{k}{T_{k}}\lambda^{1-\rho}_{k}}{\partial{T_{c'}}}\Bigg)\Bigg(\frac{T_{c'}}{\sum\limits_{k}{T_{k}}{\lambda_{k}^{1-\rho}}}\Bigg)
\end{equation*}

Notice also that:

\begin{equation*}
    \frac{\partial\lambda_{k}}{\partial{T_{c'}}} = \Bigg(\frac{1}{1-\rho}\Bigg)\Bigg(\frac{[t_{ck}(Z_{c}^{-\theta})]^{\frac{1}{1-\rho}}}{T_{c'}}\Bigg)
\end{equation*}

We can therefore solve for this cross-derivative as the following:

\begin{align*}
    \frac{\partial\ln{\pi_{c}}}{\partial\ln{T_{c'}}} & = \sum\limits_{k}\Big[-\Big(\frac{\rho}{1-\rho}\Big)\phi_{ck}{\pi_{{c'}k}}\Big]+\sum\limits_{k}\Big[-\omega_{k}\pi_{{c'}k}\Big] \\ &= -\sum\limits_{k}{\pi_{{c'}k}}\Big[\omega_{k}+\Big(\frac{\rho}{1-\rho}\Big)\phi_{ck}\Big]
\end{align*}

\subsection{Regression Tables}

\subsubsection{First Stage Shift Share Regression}
\begin{table}[!htbp] \centering
\begin{tabular}{@{\extracolsep{5pt}}lc}
\\[-1.8ex]\hline
\hline \\[-1.8ex]
& \multicolumn{1}{c}{\textit{Dependent variable: wage}} \
\cr \cline{2-2}
\\[-1.8ex] & (1) \\
\hline \\[-1.8ex]
 sim\_wage & 0.523$^{***}$ \\
& (0.037) \\
\hline \\[-1.8ex]
 Observations & 19280 \\
 $R^2$ & 0.881 \\
 Adjusted $R^2$ & 0.863 \\
 Residual Std. Error & 0.198 (df=16799) \\
 F Statistic & 3696120.683$^{***}$ (df=2480; 16799) \\
\hline
\hline \\[-1.8ex]
\textit{Note:} & \multicolumn{1}{r}{$^{*}$p$<$0.1; $^{**}$p$<$0.05; $^{***}$p$<$0.01} \\
\end{tabular}
\end{table}

\subsubsection{OLS vs IV Regression}
\begin{table}[!htbp] \centering
    \caption{Estimates Of The Effect Of Wages On City Shares With City And Occupation Fixed Effects}
    \begin{tabular}{@{\extracolsep{5pt}}lcc}
        \\[-1.8ex]\hline
        \hline                                                                                        \\[-1.8ex]
                    & \multicolumn{2}{c}{\textit{Dependent variable: City Shares}} \
        \cr \cline{2-3}
        \\[-1.8ex] & \multicolumn{1}{c}{OLS} & \multicolumn{1}{c}{IV}  \\
        \\[-1.8ex] & (1) & (2) \\
        \hline                                                                                        \\[-1.8ex]
        Wage        & 2.777$^{***}$                                                   &               \\
                    & (0.041)                                                         &               \\
                    &                                                                 & 0.291$^{***}$ \\
                    &                                                                 & (0.023)       \\
        \hline                                                                                        \\[-1.8ex]
        $R^2$       & 0.956                                                           &               \\
        F Statistic &                                                                 & 145.8$^{***}$ \\
        \hline
        \hline                                                                                        \\[-1.8ex]
        % \textit{Note:} & \multicolumn{2}{r}{$^{*}$p$<$0.1; $^{**}$p$<$0.05; $^{***}$p$<$0.01}                   \\
    \end{tabular}
    \caption*{\small\textit{Note: This is the results of the regression using the estimation equation discussed in section 3.1. All city and occupation fixed effects are omitted for brevity. Standard errors are clustered on city-occupation pairs and are in parenthesis. $^{*}$p$<$0.1; $^{**}$p$<$0.05; $^{***}$p$<$0.01}}
    \label{city_sec_iv_ols}
\end{table}

% \subsection{Cross City Elasticity}

% \begin{align*}
%     \ln \pi_c = \frac{1}{1 - \rho} \ln Z_c^{- \theta} + \ln \left( \sum_{k}^{} (T_c T_k t_{ck})^{\frac{1}{1 - \rho}} \lambda_k^{- \rho} \right) - \ln \left( \sum_{k}^{} \lambda_k^{1 - \rho} \right)
% \end{align*}

% \begin{align*}
%     \frac{\partial \ln \pi_c}{\partial \ln T_{c'}} = \frac{\partial \left( \sum_{k}^{} (T_c T_k t_{ck})^{\frac{1}{1 - \rho}} \lambda_k^{- \rho} \right)}{\partial T_{c'}} \frac{T_{c'}}{\sum_{k}^{} T_{ck}^{\frac{1}{1 - \rho}} \lambda_k^{- \rho}} - \frac{\partial \left( \sum_{k}^{} \lambda_k^{1 - \rho} \right)}{\partial T_{c'}} \frac{T_{c'}}{\sum_{k}^{} \lambda_k^{1 - \rho}}
% \end{align*}

% Consider the derivative of $\lambda_k$ with respect to $T_{c'}$:

% \begin{align*}
%     \frac{\partial \lambda_k}{\partial T_{c'}} & = T_k^{\frac{1}{1 - \rho}} \left( \frac{1}{1 - \rho} \right) T_{c'}^{\frac{1}{1 - \rho} - 1} t_{c'k}^{\frac{1}{1 - \rho}} (Z_c^{- \theta})^{\frac{\rho}{1 - \rho}} \\
%                                                & = \frac{1}{1 - \rho} \frac{1}{T_{c'}} (T_{c'k} Z_c^{- \theta})^{\frac{1}{1 - \rho}}
% \end{align*}

% Consider the first term in the derivative:

% \begin{align*}
%     \frac{\partial \left( \sum_{k}^{} (T_c T_k t_{ck})^{\frac{1}{1 - \rho}} \lambda_k^{- \rho} \right)}{\partial T_{c'}} \frac{T_{c'}}{\sum_{k}^{} T_{ck}^{\frac{1}{1 - \rho}} \lambda_k^{- \rho}} & = \frac{\sum_{k}^{} T_{ck}^{\frac{1}{1 - \rho}} \lambda_k^{-\rho - 1} \left( \frac{1}{1 - \rho} \right) \frac{1}{T_{c'}} T_{c'k}^{\frac{1}{1 - \rho}} (Z_c^{- \theta})^{\frac{1}{1 - \rho}}}{\sum_{k}^{} T_{ck}^{\frac{1}{1 - \rho}} \lambda_k^{- \rho}} \times T_{c'} \\
%                                                                                                                                                                                                    & = - \frac{\rho}{1 - \rho} \frac{\sum_{k}^{}T_{ck}^{\frac{1}{1 - \rho}} \lambda_k^{- \rho} \pi_{c'k}}{\sum_{k}^{} T_{ck}^{\frac{1}{1 - \rho}} \lambda_k^{- \rho}}                                                                                                       \\
% \end{align*}

% Now consider the second term in the derivative:

% \begin{align*}
%     - \frac{\partial \left( \sum_{k}^{} \lambda_k^{1 - \rho} \right)}{\partial T_{c'}} \frac{T_{c'}}{\sum_{k}^{} \lambda_k^{1 - \rho}} & = - \frac{- \sum_{k}^{} \rho \lambda_k^{- \rho - 1} \left( \frac{1}{1 - \rho} \right) \frac{1}{T_{c'}} (T_{c'k} Z_c^{- \theta})^{\frac{1}{1 - \rho}}}{\sum_{k}^{} \lambda_k^{- \theta}} \times T_{c'} \\
%                                                                                                                                        & = \frac{\rho}{1 - \rho} \frac{\sum_{k}^{} \lambda_k^{- \rho} \pi_{c'k} }{\sum_{k}^{} \lambda_k^{- \rho}}                                                                                              \\
% \end{align*}

% This will give us:

% \begin{align*}
%     \frac{\partial \ln \pi_c}{\partial \ln T_{c'}} = \frac{\rho}{1 - \rho} \left[ \frac{\sum_{k}^{} \lambda_k^{- \rho} \pi_{c'k}}{\sum_{k}^{} \lambda_k^{- \rho}} - \frac{\sum_{k}^{} T_{ck}^{\frac{1}{1 - \rho}} \lambda_k^{-\rho} \pi_{c'k}}{\sum_{k}^{} T_{ck}^{\frac{1}{1 - \rho}} \lambda_k^{-\rho}} \right]
% \end{align*}

% \subsection{Technology Elasticity}

% \begin{align*}
%     \frac{\partial \ln \pi_c}{\partial \ln T_k} = \frac{\partial \left( \sum_{k}^{} (T_c T_k t_{ck})^{\frac{1}{1 - \rho}} \lambda_k^{- \rho} \right)}{\partial T_k} \frac{T_k}{\sum_{k}^{} T_{ck}^{\frac{1}{1 - \rho}} \lambda_k^{- \rho}} - \frac{\partial \left( \sum_{k}^{} \lambda_k^{1 - \rho} \right)}{\partial T_k} \frac{T_k}{\sum_{k}^{} \lambda_k^{1 - \rho}}
% \end{align*}

% Consider the derivative of $\lambda_k$ with respect to $T_k$:

% \begin{align*}
%     \frac{\partial \lambda_k}{\partial T_k} & = \frac{1}{1 - \rho} T_k^{\frac{1}{1 -\rho} - 1} \sum_{k}^{} (T_c t_{ck} Z_c^{- \theta})^{\frac{1}{1 - \rho}} \\
%                                             & =\frac{1}{1 - \rho} \frac{1}{T_k} \lambda_k
% \end{align*}

% Consider the first term in the derivative:

% \begin{align*}
%     \frac{\partial \left( \sum_{k}^{} (T_c T_k t_{ck})^{\frac{1}{1 - \rho}} \lambda_k^{- \rho} \right)}{\partial T_k} \frac{T_k}{\sum_{k}^{} T_{ck}^{\frac{1}{1 - \rho}} \lambda_k^{- \rho}} & = \frac{T_k^{\frac{1}{1 - \rho}} (T_c t_{ck})^{\frac{1}{1 - \rho}} (- \rho) \lambda_k^{- \rho - 1} \left( \frac{1}{1 - \rho} \right) \frac{1}{T_k} \lambda_k + \lambda_k^{- \rho} \left( \frac{1}{1 - \rho} \right) \frac{1}{T_k} T_{ck}^{\frac{1}{1 - \rho}}}{\sum_{k}^{} T_{ck}^{\frac{1}{1 - \rho}} \lambda_k^{- \rho}} \\
%                                                                                                                                                                                              & = \phi_{ck}
% \end{align*}

% Consider the second term in the derivative:

% \begin{align*}
%     - \frac{\partial \left( \sum_{k}^{} \lambda_k^{1 - \rho} \right)}{\partial T_k} \frac{T_k}{\sum_{k}^{} \lambda_k^{1 - \rho}} & = - \frac{(1 - \rho) \lambda_k^{- \rho} \left( \frac{1}{1 - \rho} \right) \frac{1}{T_k} \lambda_k}{\sum_{k}^{} \lambda_k^{1 - \rho}} \times T_k \\
%                                                                                                                                  & = - \omega_k
% \end{align*}

% This will give us:

% \begin{align*}
%     \frac{\partial \ln \pi_c}{\partial \ln T_k} = \phi_{ck} - \omega_k
% \end{align*}

\end{document}