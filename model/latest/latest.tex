\documentclass[10pt]{article}
\usepackage[utf8]{inputenc}
\usepackage[shortlabels]{enumitem}
\usepackage[margin=1in]{geometry}
\usepackage{setspace}
\usepackage{hyperref}
\usepackage{graphicx}
\usepackage{natbib}
\onehalfspacing
\setlength{\parskip}{1em}
\setlength{\parindent}{0pt}
\bibliographystyle{ecta}
\usepackage[T1]{fontenc}
\usepackage{titling}
\usepackage{amsmath}
\setlength{\droptitle}{-7em}
\addtolength\abovedisplayskip{-3in}
\addtolength\belowdisplayskip{-3in}

\title{City Location Choice and Household Productivity}
\preauthor{}
\postauthor{}
\author{Tan Sein Jone}
\predate{}
\postdate{}
\date{}
%\Today

\begin{document}
\doublespacing
\maketitle

\section{Introduction}

Cities capture a disproportionate amount of household location choices in the world. In 2018, 55\% of the world's population lived in urban areas, and this is expected to increase to 68\% by 2050. This is because of the benefits cities provide, such as access to jobs, amenities, and social networks. These benefits are often thought of as universal to all households, regardless of their occupation specific skills. However, the benefits of every city is clearly not the same for every household. An auto worker from Detroit will not view New York as an equivalent substitute for a city to live in as a finance worker from New York. Given the billions of dollars spent on place based policies, it is important for us to separate out this dimension of comparative advantage, where a city is able to provide higher levels of productivity for households working in specific occupations.

A city's comparative advantage exists in three dimensions. First, there is a city specific advantage which makes all occupations more productive in that city. This is due to the city's size, more specifically it's population which results in better amenities, more jobs, and a larger social network. Second, there is an occupation specific advantage which is universal across all cities. Some jobs are simply more productive than others and cities that specialize in that job will be observed as having a nigher overall choice shares of households. Think cities that specialize in highly attractive occupations such as professional services. Finally, there is a city-occupation specific advantage. A city with a comparative advantage in this dimension may have specific policies that make it more attractive for a specific occupation such as tax breaks for mechanics in Chicago. One crucial distinction we need to make is between occupations and sectors. Sectors are broad categories of jobs such as manufacturing, services, and agriculture. Occupations are specific jobs such as mechanic and clerk. Cities can specialize in occupations by specializing in sectors that most intensely use that occupation.

In this paper, we propose a model of city location choices based on a class of models in the quantitative trade literature first proposed by \cite{ek}. These models treat productivity as random draws from Fr\'{e}chet distributions, allowing for a non zero production of goods which will help us to account for small cities. \cite{redding} applies this model to an urban setting by loading city heterogeneity onto amenities, capturing the first dimension of comparative advantage. Both these models however rely on the independence assumption which treats that goods and cities are perfect substitutes for each other. \cite{lindandramondo} break this assumption by introducing a nested CES utility structure which allows for correlated draws of productivity within nests across countries. We will apply this structure to the urban setting by treating occupations as independent nests. This will allow us to capture substitution patterns where cities that have a similar occupational structure will be treated as better substitutes for each other. By doing this, we are able to capture the second and third dimensions of comparative advantage.

We start in the first section by adapting the model of \cite{lindandramondo} to the urban setting. Unlike their original paper where the nests are implied, we will explicitly model the nests as occupations. A few implications will result from this specification. First, we will be able to observe both overall city shares and city shares specific specific to each occupation. As well as city occupation compositions which can be used as a measure of similarity between cities. Second, using this data, we will derive elasticities for all three dimensions of comparative advantage, that being the city specific, occupation specific and city-occupation specific elasticities. While the last elasticity will serve as a sanity check for the direction of substitution. City and occupation elasticities will be used to verify our claim of unbalanced substitution patterns that favor similar cities. In other words, shocks to cities of a certain occupational composition will have larger effects of cities with similar compositions due to correlations in productivity draws.

In the second section, we will use data from the American Community Survey which has on the metropolitan area and occupational level to present some stylized facts about cities. We will then be presenting our estimation equations that will be used to recover our parameters. Once the model is calibrated, we will use it to run counterfactuals and simulate the effects of a policy at both the city and occupation level. This is where it is possible for us to evaluate the effectiveness of place based policies and also where we will make statements about how these policies should be designed to maximize their effectiveness.

\section{Model}

\begin{align*}
    Y_{ck} = Z_{ck}
\end{align*}

Consider a closed economy consisting of $N$ cities and $K$ occupations. Each city $c$ employs households in occupation $k$ to produce output $Y_{ck}$. We assume no trade costs between cities and that the price of the good produced is freely traded and priced under perfect competition. The price index of city $c$ is given by $\Phi_c$ and the wage of a household living in city $c$ working in occupation $k$ is given by $w_{ck} = \Phi_c Z_{ck}$ where $Z_{ck}$ is the productivity of the household. We assume that when a household chooses to work in a particular occupation, they will be randomly allocated to a sector within a given city. A household has symmetric CES preferences over cities and occupations and will choose a city-occupation pair that maximizes their utility. We will assume that households are perfectly mobile across cities and that there are no fixed costs to moving. The productivity of a household within a given city-occupation is drawn from a Fr\'{e}chet distribution. In \cite{ek}, the Fr\'{e}chet distribution is used to model productivity draws in a trade setting and has two main components. The shape parameter $\theta$ which determines the dispersion of draws across the distribution and reflects the heterogeneity of cities. $\theta$ will also determine the gains from city choices, ie how much households will gain from having more choices of cities to live in, which is analogous to gains from trade. The scale parameter $T_{ck}$ is a level shifter that determines the absolute advantage of a city-occupation pair, reflecting its attractiveness. Our specification so far assumes independent draws of productivity across cities, the joint distribution of which is given by:

\begin{align*}
    P[Z_1 < z_1, \dots, Z_N < z_N] & = \prod_{c}^{N} P[Z_c < z_c]                                 \\
                                   & = \exp \left\{ - \sum_{c}^{N} T_{ck} Z_c^{- \theta} \right\}
\end{align*}

This result implies that the choice shares of households for a given city is equal to the probability of households choosing that city. So far, we have stuck with the the base EK specification with the only modification being the mapping of productivity onto wages rather than prices which allows us to obtain city choice shares. We will now abandon the independence assumption and introduce a structure that allows for correlated draws across cities.

\begin{align*}
    P[Z_1 < z_1, \dots, Z_N < z_N] = \exp \left\{ - \left( \sum_{c}^{N} (T_{ck} Z_c^{- \theta})^{\frac{1}{1 - \rho}} \right)^{1 - \rho} \right\}
\end{align*}

Notice with this structure that the shape of the distribution is no longer purely determined by $\theta$ but also by the correlation parameter $\rho$. Where the distribution of productivity within a given occupation decreases as $\rho$ increases. Meaning that within a given occupation, the higher correlation will result in cities being more similar with each other in terms of productivity. When $\rho = 0$, the distribution reduces down to the original joint distribution with the shape being purely determined by $\theta$ and where draws are independent. As $\rho \rightarrow 1$, the draws become perfectly correlated and the productivity is identical across all cities.

\subsection{Cross Nested CES}

\begin{align}
    P[Z_1(\nu) < z, \dots, Z_N(\nu) < z] = \exp \left\{ - \sum_{k}^{K} \left[ \sum_{c}^{N} (T_{ck} Z_c^{- \theta})^{\frac{1}{1 - \rho_k}} \right]^{1 - \rho_k} \right\}
\end{align}

We will assume that productivity is distributed max stable multivariate Fr\'{e}chet with $T_{ck}$ being the scale parameter for city $c$ and occupation $k$. $\theta > 0$ is the shape parameter that determines the dispersion of draws across the distribution and $\rho_k$ is the occupation specific correlation parameter. As in \cite{lindandramondo}, we adopt a cross nested CES structure with a correlational structure within independent occupational nests. This has the characteristic of distributions not purely determined by $\theta$ but also by the correlation parameter. This is because the correlation parameter determines the degree of correlation between draws of productivity across cities.

In order to identify the three dimensions of comparative advantage, we will be separating out the scale parameter into three components where $T_{ck} = T_c T_k t_{ck}$. $T_c$ is the city specific scale parameter which captures the attractiveness of a city, shifting the distribution of all productivity draws within that city upwards. $T_k$ is the occupation specific scale parameter which is common across all cities, this captures the effect that any occupation has across all cities. $t_{ck}$ is the city-occupation specific scale parameter which captures the effect of a city on a specific occupation. This will give us the final dimension of comparative advantage.

\begin{align}
    Z = \max_c \left\{ \frac{Z_c}{\Phi_c} \right\}
\end{align}

A household's schedule of productivity is characterized by a vector of draws from different Fr\'{e}chet distributions for each city occupation pair. The realized productivity a household has in city $c$ is the occupation that maximizes the productivity has in the city. Unlike sequential games where households might pick a city before picking an occupation, we assume that households pick both simultaneously. A way to contextualize this is to think of households as already having an ideal occupation in mind when picking a city, hence $Z_c = \max_k \left\{ Z_{ck} \right\}$. This productivity is then scaled by the price index of a city to reflect how high prices reduces the purchasing power of the household within that city, hence making it less attractive.

\subsection{Equilibrium}

\begin{align}
    G(Z_1^{- \theta}, \dots, Z_N^{- \theta}) = \sum_{k}^{K} T_k \left[ \sum_{c}^{N} (t_{ck} Z_c^{- \theta})^{\frac{1}{1 - \rho_k}} \right]^{1 - \rho_k}
    \label{nested_ces}
\end{align}

We first define $Z_c^{- \theta} = (\gamma \Phi_c)^{- \theta} T_c$ and $\gamma = \Gamma (\frac{\theta - 1}{\theta})$. Given our definition of $Z_c^{- \theta}$, $T_c$ will be absorbed within $Z_c^{- \theta}$, giving us the correlation function $G(Z_1^{- \theta}, \dots, Z_N^{- \theta})$ as defined in equation \ref{nested_ces}.

\begin{align*}
    \pi_c = \frac{Z_c^{- \theta} G_c(Z_1^{- \theta}, \dots, Z_N^{- \theta})}{G(Z_1^{- \theta}, \dots, Z_N^{- \theta})}
\end{align*}

Where $G_c (Z_1^{- \theta}, \dots, Z_N^{- \theta}) = \partial G(Z_1^{- \theta}, \dots, Z_N^{- \theta}) / \partial Z_c^{- \theta}$. The expression above gives us the city specific choice shares of households across all occupations. In order to evaluate the expression, we make the simplifying assumption that the correlation parameter is the same across all occupations $\rho_k = \rho$.

\begin{align*}
    \lambda_{k} = \sum_{c}^{N} \left( t_{ck} Z_{c}^{-\theta} \right)^{\frac{1}{1-\rho}}
\end{align*}

We introduce the above expression which measures the appeal of occupation $k$. Given that the expression does not include $T_k$, it does not capture the attractiveness of occupation $k$. Instead the expression captures the extent to which a given occupation $k$ exhibits city-specific productivities which are correlated with aggregate city-level attractiveness. The following definitions will be useful in simplifying the expression for city shares.

\begin{align*}
    \pi_{ck} = \frac{Z_c^{- \theta} G_c^k(Z_1^{- \theta}, \dots, Z_N^{- \theta})}{G^k(Z_1^{- \theta}, \dots, Z_N^{- \theta})}
\end{align*}

To obtain occupation specific choice shares, we can evaluate the derivative for city shares at the occupation level. More specifically, rather than $G(Z_1^{- \theta}, \dots, Z_N^{- \theta})$, we will be evaluating $G^k(Z_1^{- \theta}, \dots, Z_N^{- \theta}) = \sum_{c}^{N} (t_{ck} Z_c^{- \theta})^{\frac{1}{1 - \rho_k}}$. This will give us the following:

\begin{align}
    \pi_{ck} = \frac{(t_{ck} Z_c^{-\theta})^{\frac{1}{1 - \rho}}}{\lambda_k}
    \label{city_occuaption_shares}
\end{align}

Moving forward, it will be convenient for us to provide the following definitions. First, we define $\phi_{ck}$ as the comparative advantage of city $c$ in occupation $k$. This can be interpreted as an occupation's within city choice shares. As such $\phi_{ck}$ will be between 0 and 1. We define this formally as:

\begin{align*}
    \phi_{ck} \equiv \frac{{t^{\frac{1}{1-\rho}}_{ck}}{T_{k}}\lambda_{k}^{-\rho}}{\sum\limits_{k}{t^{\frac{1}{1-\rho}}_{ck}}{T_{k}}\lambda_{k}^{-\rho}}
\end{align*}

Second, we define an occupation specific choice share which takes into account both the size of each occupation, as measured by productivity $T_k$, and the extent to which individual city-occupation productivities are correlated with overall city attractiveness. This is defined as the following expression:

\begin{align*}
    \omega_k \equiv \frac{{T_{k}}\lambda_{k}^{1-\rho}}{\sum\limits_{k}{T_{k}}\lambda_{k}^{1-\rho}}
\end{align*}

From the perspective of discrete choice demand models, $\omega_k$ can be interpreted as the "market share" of occupation $k$. This is because it captures the extent to which the productivity of occupation $k$ is correlated with overall city attractiveness. The expression is normalized by the sum of all $\omega_k$ to ensure that the sum of all market shares is equal to 1.

\begin{align}
    \frac{\pi_{ck}}{\pi_c}\equiv \frac{\phi_{ck}}{\omega_k}
    \label{identity}
\end{align}

The expression above tells us the extent to which a given occupation $k$ is concentrated in city $c$ above and beyond the predicted concentration, given the size of the city $\pi_c$ is equivalent to the within-city choice share in occupation $k$ divided by the aggregate choice share of that occupation.

\begin{align*}
    \pi_{c} = \sum_{k}^{K} \frac{(Z_{c}^{-\theta} t_{ck})^{\frac{1}{1-\rho}}}{\lambda_k} \frac{T_k \lambda_{k}^{1 - \rho}}{\sum \limits_{k}^{K} T_k \lambda_{k}^{1-\rho}}
\end{align*}

\begin{align}
    \pi_c = \sum_{k}^{K} \pi_{ck} \omega_k
\end{align}

By using our definitions, we arrive at the above expression for city shares, which is now the sum of all occupation specific shares weighted by the market share of each occupation.

\subsection{City-Occupation Elasticities}

We now turn to the city-occupation derivatives: city $c$'s own technology elasticity and cross technology elasticity with respect to another city $c'$. This derivative will be approximated by assuming that each city is small, and therefore $\partial \ln \lambda_k / \partial \ln t_{ck} \approx 0$. That is, the aggregate "price index" associated with location choice in occupation $k$ is not affected by a marginal change in the productivity of a given city-occupation pair. We then obtain the following for the own technology shock:

\begin{align}
    \frac{\partial \ln \pi_{ck}}{\partial \ln t_{ck}} \approx \phi_{ck} \left( \frac{1}{1 - \rho} \right)
    \label{co_own_elasticity}
\end{align}

This elasticity is positive and increasing in a city's comparative advantage in occupation $k$. The intuition is that a city that is more attractive for occupation $k$ will see a larger increase in the choice share of that occupation in response to a marginal increase in the productivity of that occupation. The more specialized the city is in that occupation, the larger the effect of a productivity shock in that occupation. As $\rho \rightarrow 1$, the elasticity approaches infinity, as within any given occupation, if they become more correlated, cities within that given occupation become perfect substitutes for each other.

\begin{align}
    \frac{\partial\ln{\pi_{c}}}{\partial\ln{t_{{c'}k}}} = -{\pi_{c'}}{\phi_{ck}}\Big[1+\Big(\frac{\rho}{1-\rho}\Big)\Big(\frac{\pi_{ck}}{\pi_{c}}\Big)\Big]
    \label{co_cross_elasticity}
\end{align}

Turning our attention to the cross city technology elasticity, we see that the first term captures the CES element of consumer choice. As the city $c'$ increases in size and becomes increasingly competitive in that occupation, any shocks to this city-occupation pair will have an increasingly negative effects on all other cities.

How much the other cities will be affected is captured by the second term which captures the unbalanced substitution patterns. If a city is relatively concentrated in occupation $k$, then the affects of a shock in that given occupation will be felt more severely, this is shown by the ratio $\pi_{ck} / \pi_c$. Notice also that if $\rho = 0$, this elasticity simplified to the standard CES elasticity. As $\rho \rightarrow 1$, the elasticity approaches infinity and the correlated choices start to dominate.

\subsection{Aggregate Elasticities}

We now consider the elasticity of city shares with respect to aggregate city technology shocks $T_c$ and aggregate occupation shocks $T_k$. Beginning with the former, if we assume again that each cit is small and that $\partial \ln \lambda_k / \partial \ln T_c \approx 0$, we get the following for a city's own technology shock:

\begin{align*}
    \frac{\partial \ln \pi_c}{\partial \ln T_c} \approx \frac{1}{1 - \rho}
\end{align*}

This is simply the summation of city-occupation elasticities which was derived in expression \ref{co_own_elasticity}. And given that by definition $\sum \limits_k^K \phi_{ck} = 1$, we get the following derivations:

\begin{align*}
    \frac{\partial{\ln{\pi_{c}}}}{\partial\ln{T_{c}}} & = \sum\limits_{k}\frac{\partial{\ln{\pi_{c}}}}{\partial\ln{t_{ck}}} \\ &= \sum\limits_{k}\phi_{ck}\Bigg(\frac{1}{1-\rho}\Bigg) \\ &= \Bigg(\frac{1}{1-\rho}\Bigg)\sum\limits_{k}\phi_{ck} = \frac{1}{1-\rho}
\end{align*}

The result tells us that if $\rho = 0$ and there's no correlation between occupations, then the elasticity is simply 1. We get a similar story when deriving the cross city technology elasticity. Using the same assumption, we arrive at the result:

\begin{align}
    \frac{\partial\ln{\pi_{c}}}{\partial\ln{T_{c'}}} = -\pi_{c'}\Bigg[\sum\limits_{k}\phi_{{c'}k}\Big[1+\Big(\frac{\rho}{1-\rho}\Big)\Big(\frac{\pi_{ck}}{\pi_{c}}\Big)\Big]\Bigg]
\end{align}

Similarly, this is simply the sum over all city-occupation elasticities derived in expression \ref{co_cross_elasticity}. This is shown in the following derivation:

\begin{align*}
    \frac{\partial\ln{\pi_{c}}}{\partial\ln{T_{c'}}} & = \sum\limits_{k}\frac{\partial\ln{\pi_{c}}}{\partial\ln{t_{{c'}k}}} \\ &= \sum\limits_{k}\Bigg[-{\pi_{c'}}{\phi_{ck}}\Big[1+\Big(\frac{\rho}{1-\rho}\Big)\Big(\frac{\pi_{ck}}{\pi_{c}}\Big)\Big]\Bigg]\\ &= -\pi_{c'}\Bigg[\sum\limits_{k}\phi_{{c'}k}\Big[1+\Big(\frac{\rho}{1-\rho}\Big)\Big(\frac{\pi_{ck}}{\pi_{c}}\Big)\Big]\Bigg]
\end{align*}

The interpretation of this elasticity is the same as expression \ref{co_cross_elasticity}. The first term captures the CES element of consumer choice and the second term captures the unbalanced substitution patterns, with $\rho$ governing the strength of this unbalanced substitution. Now we simply sum over all occupations to get the aggregate elasticities.

Lastly, we derive the elasticity f city shares with respect to aggregate occupation shocks $T_k$. This is fairly straightforward as wel can simply evaluate the following:

\begin{align*}
    \frac{\partial\ln{\pi_{c}}}{\partial\ln{T_{k}}} & = \Big(\frac{\partial{\phi_{ck}}}{\partial{T_{k}}}\Big)\Big(\frac{T_{k}}{\phi_{ck}}\Big) - \Big(\frac{\partial{\omega_{k}}}{\partial{T_{k}}}\Big)\Big(\frac{T_{k}}{\omega_{k}}\Big) \\ &= \phi_{ck}-\omega_{k}
\end{align*}

By using the identity in expression \ref{identity}, we can rewrite the elasticity as the following:

\begin{equation}
    \frac{\partial\ln{\pi_{c}}}{\partial\ln{T_{k}}} = {\omega_{k}}\Bigg[\frac{\pi_{ck}}{\pi_{c}}-1\Bigg]
\end{equation}

You should first notice that the second term in brackets can be either positive or negative depending on the city's relative concentration in that occupation. If the city is relatively concentrated and $\pi_{ck} / \pi_c > 1$, then the city will face a negative shock. If the city is relatively less concentrated and $\pi_{ck} / \pi_c < 1$, then the city will face a negative shock. We can think of this as certain occupations start to become more attractive, cities that are less concentrated in that occupation will face a negative shock as households start to move to cities that are more concentrated in that occupation. This second term is scaled by the market share of that occupation, $\omega_k$, which is simply the market share of that occupation across all cities.

\newpage
\bibliography{latest}

\newpage

\section{Appendix}

\subsection{Derivations}

\noindent\textbf{City Choice Shares $\pi_{c}$}: As discussed in Lind and Ramondo (2023), the choice shares can be expressed as:

\begin{equation*}
    \pi_{c}=\frac{Z_{c}^{-\theta}{G_{c}}}{G()}
\end{equation*}

We define $G()$ as the following, given the definition of $T_{ck}$ and $\lambda_{k}$ provided in the main text:

\begin{equation*}
    G({Z_{1}}^{-\theta},...,{Z_{N}}^{-\theta})=\sum\limits_{k}\Big[\sum\limits_{c}^{N}({T_{ck}}{Z_{c}^{-\theta}})^{\frac{1}{1-\rho_{k}}}\Big]^{1-\rho_{k}} = \sum\limits_{k}{T_{k}}\lambda^{1-\rho}_{k}
\end{equation*}

$G_{c}()$ is simply the derivative of $G()$ with respect to $Z_{c}^{-\theta}$.

\begin{align*}
    \frac{\partial{G()}}{\partial{Z_{c}^{-\theta}}} & = \sum\limits_{k}\frac{\partial}{\partial{Z_{c}^{-\theta}}}\Big[\sum\limits_{c}^{N}({T_{ck}}{Z_{c}^{-\theta}})^{\frac{1}{1-\rho}}\Big]^{1-\rho} \\ &= \sum\limits_{k}\frac{\partial}{\partial{Z_{c}^{-\theta}}}\Big[{T^{\frac{1}{1-\rho}}_{k}}\sum\limits_{c}^{N}({t_{ck}}{Z_{c}^{-\theta}})^{\frac{1}{1-\rho}}\Big]^{1-\rho} \\ &= \sum\limits_{k}{T_{k}}\Big[(1-\rho)\lambda^{-\rho}_{k}\frac{\partial{\lambda_{k}}}{\partial{Z_{c}^{-\theta}}}\Big] \\ &= \sum\limits_{k}{T_{k}}\Big[(1-\rho)\lambda^{-\rho}_{k}(\frac{1}{1-\rho}){t^{\frac{1}{1-\rho}}_{ck}}(Z_{c}^{-\theta})^{\frac{\rho}{1-\rho}}\Big]\\ &= (Z_{c}^{-\theta})^{\frac{\rho}{1-\rho}}\sum\limits_{k}{T_{k}}{t^{\frac{1}{1-\rho}}_{ck}}\lambda_{k}^{-\rho}
\end{align*}

Putting these three terms together, we derive the choice shares:

\begin{align*}
    \pi_{c} & = \frac{Z_{c}^{-\theta}{G_{c}}}{G()} \\ &= (Z_{c}^{-\theta})^{\frac{1}{1-\rho}}\Bigg[\frac{\sum\limits_{k}{t^{\frac{1}{1-\rho}}_{ck}}{T_{k}}\lambda_{k}^{-\rho}}{\sum\limits_{k}{T_{k}}\lambda_{k}^{1-\rho}}\Bigg]
\end{align*}

\noindent\textbf{Own Occupation-Specific Elasticity:} We wish to evaluate $\partial\ln{\pi_{c}}/\partial\ln{t_{ck}}$, under the assumption that $\partial\ln{\lambda_{k}}/\partial\ln{t_{ck}}=0$. Notice this assumption yields the following:

\begin{equation*}
    \frac{\partial\ln{\pi_{c}}}{\partial\ln{t_{ck}}} \approx \frac{\partial\ln{\pi_{ck}}}{\partial\ln{t_{ck}}} - \frac{\partial\ln{\phi_{ck}}}{\partial\ln{t_{ck}}}
\end{equation*}

given the definitions of $\pi_{ck}$ and $\phi_{ck}$ provided in the text.

Notice that this only leaves one term to evaluate if we take the logarithm of $\pi_{c}$ and the derivative of this logarithm with respect to $\ln{t_{ck}}$:

\begin{align*}
    \frac{\partial\ln{\pi_{c}}}{\partial\ln{t_{ck}}} & \approx \frac{\partial\ln[{\sum\limits_{k}{t^{\frac{1}{1-\rho}}_{ck}}{T_{k}}\lambda_{k}^{-\rho}}]}{\partial\ln{t_{ck}}} \\ &= \Bigg(\frac{\partial[{\sum\limits_{k}{t^{\frac{1}{1-\rho}}_{ck}}{T_{k}}\lambda_{k}^{-\rho}}]}{\partial{t_{ck}}}\Bigg)\Bigg(\frac{t_{ck}}{{\sum\limits_{k}{t^{\frac{1}{1-\rho}}_{ck}}{T_{k}}\lambda_{k}^{-\rho}}}\Bigg)\\ &= \Bigg(\frac{1}{1-\rho}\Bigg)\Bigg(\frac{t_{ck}^{\frac{1}{1-\rho}}{T_{k}}\lambda_{k}^{-\rho}}{t_{ck}}\Bigg)\Bigg(\frac{t_{ck}}{{\sum\limits_{k}{t^{\frac{1}{1-\rho}}_{ck}}{T_{k}}\lambda_{k}^{-\rho}}}\Bigg) \\ &= \Big(\frac{1}{1-\rho}\Big)\Bigg[\frac{{t^{\frac{1}{1-\rho}}_{ck}}{T_{k}}\lambda_{k}^{-\rho}}{\sum\limits_{k}{t^{\frac{1}{1-\rho}}_{ck}}{T_{k}}\lambda_{k}^{-\rho}}\Bigg]\\ &= \frac{\phi_{ck}}{1-\rho}
\end{align*}

\noindent\textbf{Cross-City Occupation-Specific Elasticity:} We wish to evaluate $\partial\ln{\pi_{c}}/\partial\ln{t_{{c'}k}}$. Notice that this elasticity can be expressed as the following two terms:

\begin{equation*}
    \frac{\partial\ln{\pi_{c}}}{\partial\ln{t_{{c'}k}}} = {t^{\frac{1}{1-\rho}}_{ck}}{T_{k}}\Big(\frac{\partial\lambda_{k}^{-\rho}}{\partial{t_{{c'}k}}}\Big)\Big(\frac{t_{{c'}k}}{{\sum\limits_{k}{t^{\frac{1}{1-\rho}}_{ck}}{T_{k}}\lambda_{k}^{-\rho}}}\Big) - {T_{k}}\Big(\frac{\partial\lambda_{k}^{1-\rho}}{\partial{t_{{c'}k}}}\Big)\Big(\frac{t_{{c'}k}}{{\sum\limits_{k}{T_{k}}\lambda_{k}^{1-\rho}}}\Big)
\end{equation*}

It will be useful to first define the derivative of $\lambda_{k}$ with respect to our variable of interest, $t_{{c'}k}$:

\begin{equation*}
    \frac{\partial{\lambda_{k}}}{\partial{t_{{c'}k}}} = \Big(\frac{1}{1-\rho}\Big)\Big(\frac{1}{t_{{c'}k}}\Big)[{t_{{c'}k}}(Z_{c'}^{-\theta})]^{\frac{1}{1-\rho}}
\end{equation*}

We can now use chain rule in order to evaluate our elasticity of interest. The first term becomes the following:

\begin{align*}
    {t^{\frac{1}{1-\rho}}_{ck}}{T_{k}}\Big(\frac{\partial\lambda_{k}^{-\rho}}{\partial{t_{{c'}k}}}\Big)\Big(\frac{t_{{c'}k}}{{\sum\limits_{k}{t^{\frac{1}{1-\rho}}_{ck}}{T_{k}}\lambda_{k}^{-\rho}}}\Big) & = {t^{\frac{1}{1-\rho}}_{ck}}{T_{k}}(-\rho)(\lambda_{k}^{-\rho-1})\Big(\frac{1}{1-\rho}\Big)\Bigg(\frac{[{t_{{c'}k}}(Z_{c'}^{-\theta})]^{\frac{1}{1-\rho}}}{{\sum\limits_{k}{t^{\frac{1}{1-\rho}}_{ck}}{T_{k}}\lambda_{k}^{-\rho}}}\Bigg) \\ &= -\Bigg(\frac{\rho}{1-\rho}\Bigg)\Bigg(\frac{[{t_{{c'}k}}(Z_{c'}^{-\theta})]^{\frac{1}{1-\rho}}}{\lambda_{k}}\Bigg)\Bigg(\frac{t^{\frac{1}{1-\rho}}_{ck}{T_{k}}{\lambda^{-\rho}_{k}}}{{\sum\limits_{k}{t^{\frac{1}{1-\rho}}_{ck}}{T_{k}}\lambda_{k}^{-\rho}}}\Bigg)\\ &= -\Bigg(\frac{\rho}{1-\rho}\Bigg)\Bigg(\frac{[{t_{{c'}k}}(Z_{c'}^{-\theta})]^{\frac{1}{1-\rho}}}{\lambda_{k}}\Bigg){\phi_{ck}}\\ &= -\Big(\frac{\rho}{1-\rho}\Big){\pi_{{c'}k}}{\phi_{ck}}
\end{align*}

The second term can be derived in the following way:

\begin{align*}
    - {T_{k}}\Big(\frac{\partial\lambda_{k}^{1-\rho}}{\partial{t_{{c'}k}}}\Big)\Big(\frac{t_{{c'}k}}{{\sum\limits_{k}{T_{k}}\lambda_{k}^{1-\rho}}}\Big) & = - {T_{k}}(\lambda_{k}^{-\rho})[{t_{{c'}k}}(Z_{c'}^{-\theta})]^{\frac{1}{1-\rho}}\Big(\frac{1}{{\sum\limits_{k}{T_{k}}\lambda_{k}^{1-\rho}}}\Big) \\ &= - \Bigg(\frac{[{t_{{c'}k}}(Z_{c'}^{-\theta})]^{\frac{1}{1-\rho}}}{\lambda_{k}}\Bigg)\Big(\frac{{T_{k}}{\lambda_{k}^{1-\rho}}}{{\sum\limits_{k} T_k \lambda_{k}^{1-\rho}}}\Big)\\ &= -\pi_{{c'}k}{\omega_{k}}
\end{align*}

Putting these together, we derive the cross-city occupation-specific elasticity as the following:

\begin{equation*}
    \frac{\partial\ln{\pi_{c}}}{\partial\ln{t_{{c'}k}}} = -{\pi_{{c'}k}}\Big[\omega_{k}+\Big(\frac{\rho}{1-\rho}\Big)\phi_{ck}]
\end{equation*}

Notice that this is equivalent to the following, given the identity linking $\pi_{c}$, $\pi_{ck}$, $\phi_{ck}$, and $\omega_{k}$.

\begin{equation}
    \frac{\partial\ln{\pi_{c}}}{\partial\ln{t_{{c'}k}}} = -{\pi_{c'}}{\phi_{ck}}\Big[1+\Big(\frac{\rho}{1-\rho}\Big)\Big(\frac{\pi_{ck}}{\pi_{c}}\Big)\Big]
\end{equation}

\noindent\textbf{Cross-City Aggregate Elasticity:} We wish to evaluate $\partial\ln{\pi_{c}}/\partial\ln{T_{c'}}$. Notice that:

\begin{equation*}
    \frac{\partial\ln{\pi_{c}}}{\partial\ln{T_{c'}}} = \Bigg(\frac{\sum\limits_{k}{{t^{\frac{1}{1-\rho}}_{ck}}}{T_{k}}{\lambda^{-\rho}_{k}}}{\partial{T_{c'}}}\Bigg)\Bigg(\frac{T_{c'}}{\sum\limits_{k}t^{\frac{1}{1-\rho}}_{ck}{T_{k}}{\lambda^{-\rho}_{k}}}\Bigg) - \Bigg(\frac{\partial\sum\limits_{k}{T_{k}}\lambda^{1-\rho}_{k}}{\partial{T_{c'}}}\Bigg)\Bigg(\frac{T_{c'}}{\sum\limits_{k}{T_{k}}{\lambda_{k}^{1-\rho}}}\Bigg)
\end{equation*}

Notice also that:

\begin{equation*}
    \frac{\partial\lambda_{k}}{\partial{T_{c'}}} = \Bigg(\frac{1}{1-\rho}\Bigg)\Bigg(\frac{[t_{ck}(Z_{c}^{-\theta})]^{\frac{1}{1-\rho}}}{T_{c'}}\Bigg)
\end{equation*}

We can therefore solve for this cross-derivative as the following:

\begin{align*}
    \frac{\partial\ln{\pi_{c}}}{\partial\ln{T_{c'}}} & = \sum\limits_{k}\Big[-\Big(\frac{\rho}{1-\rho}\Big)\phi_{ck}{\pi_{{c'}k}}\Big]+\sum\limits_{k}\Big[-\omega_{k}\pi_{{c'}k}\Big] \\ &= -\sum\limits_{k}{\pi_{{c'}k}}\Big[\omega_{k}+\Big(\frac{\rho}{1-\rho}\Big)\phi_{ck}\Big]
\end{align*}

\subsection{Cross City Elasticity}

\begin{align*}
    \ln \pi_c = \frac{1}{1 - \rho} \ln Z_c^{- \theta} + \ln \left( \sum_{k}^{} (T_c T_k t_{ck})^{\frac{1}{1 - \rho}} \lambda_k^{- \rho} \right) - \ln \left( \sum_{k}^{} \lambda_k^{1 - \rho} \right)
\end{align*}

\begin{align*}
    \frac{\partial \ln \pi_c}{\partial \ln T_{c'}} = \frac{\partial \left( \sum_{k}^{} (T_c T_k t_{ck})^{\frac{1}{1 - \rho}} \lambda_k^{- \rho} \right)}{\partial T_{c'}} \frac{T_{c'}}{\sum_{k}^{} T_{ck}^{\frac{1}{1 - \rho}} \lambda_k^{- \rho}} - \frac{\partial \left( \sum_{k}^{} \lambda_k^{1 - \rho} \right)}{\partial T_{c'}} \frac{T_{c'}}{\sum_{k}^{} \lambda_k^{1 - \rho}}
\end{align*}

Consider the derivative of $\lambda_k$ with respect to $T_{c'}$:

\begin{align*}
    \frac{\partial \lambda_k}{\partial T_{c'}} & = T_k^{\frac{1}{1 - \rho}} \left( \frac{1}{1 - \rho} \right) T_{c'}^{\frac{1}{1 - \rho} - 1} t_{c'k}^{\frac{1}{1 - \rho}} (Z_c^{- \theta})^{\frac{\rho}{1 - \rho}} \\
                                               & = \frac{1}{1 - \rho} \frac{1}{T_{c'}} (T_{c'k} Z_c^{- \theta})^{\frac{1}{1 - \rho}}
\end{align*}

Consider the first term in the derivative:

\begin{align*}
    \frac{\partial \left( \sum_{k}^{} (T_c T_k t_{ck})^{\frac{1}{1 - \rho}} \lambda_k^{- \rho} \right)}{\partial T_{c'}} \frac{T_{c'}}{\sum_{k}^{} T_{ck}^{\frac{1}{1 - \rho}} \lambda_k^{- \rho}} & = \frac{\sum_{k}^{} T_{ck}^{\frac{1}{1 - \rho}} \lambda_k^{-\rho - 1} \left( \frac{1}{1 - \rho} \right) \frac{1}{T_{c'}} T_{c'k}^{\frac{1}{1 - \rho}} (Z_c^{- \theta})^{\frac{1}{1 - \rho}}}{\sum_{k}^{} T_{ck}^{\frac{1}{1 - \rho}} \lambda_k^{- \rho}} \times T_{c'} \\
                                                                                                                                                                                                   & = - \frac{\rho}{1 - \rho} \frac{\sum_{k}^{}T_{ck}^{\frac{1}{1 - \rho}} \lambda_k^{- \rho} \pi_{c'k}}{\sum_{k}^{} T_{ck}^{\frac{1}{1 - \rho}} \lambda_k^{- \rho}}                                                                                                       \\
\end{align*}

Now consider the second term in the derivative:

\begin{align*}
    - \frac{\partial \left( \sum_{k}^{} \lambda_k^{1 - \rho} \right)}{\partial T_{c'}} \frac{T_{c'}}{\sum_{k}^{} \lambda_k^{1 - \rho}} & = - \frac{- \sum_{k}^{} \rho \lambda_k^{- \rho - 1} \left( \frac{1}{1 - \rho} \right) \frac{1}{T_{c'}} (T_{c'k} Z_c^{- \theta})^{\frac{1}{1 - \rho}}}{\sum_{k}^{} \lambda_k^{- \theta}} \times T_{c'} \\
                                                                                                                                       & = \frac{\rho}{1 - \rho} \frac{\sum_{k}^{} \lambda_k^{- \rho} \pi_{c'k} }{\sum_{k}^{} \lambda_k^{- \rho}}                                                                                              \\
\end{align*}

This will give us:

\begin{align*}
    \frac{\partial \ln \pi_c}{\partial \ln T_{c'}} = \frac{\rho}{1 - \rho} \left[ \frac{\sum_{k}^{} \lambda_k^{- \rho} \pi_{c'k}}{\sum_{k}^{} \lambda_k^{- \rho}} - \frac{\sum_{k}^{} T_{ck}^{\frac{1}{1 - \rho}} \lambda_k^{-\rho} \pi_{c'k}}{\sum_{k}^{} T_{ck}^{\frac{1}{1 - \rho}} \lambda_k^{-\rho}} \right]
\end{align*}

\subsection{Technology Elasticity}

\begin{align*}
    \frac{\partial \ln \pi_c}{\partial \ln T_k} = \frac{\partial \left( \sum_{k}^{} (T_c T_k t_{ck})^{\frac{1}{1 - \rho}} \lambda_k^{- \rho} \right)}{\partial T_k} \frac{T_k}{\sum_{k}^{} T_{ck}^{\frac{1}{1 - \rho}} \lambda_k^{- \rho}} - \frac{\partial \left( \sum_{k}^{} \lambda_k^{1 - \rho} \right)}{\partial T_k} \frac{T_k}{\sum_{k}^{} \lambda_k^{1 - \rho}}
\end{align*}

Consider the derivative of $\lambda_k$ with respect to $T_k$:

\begin{align*}
    \frac{\partial \lambda_k}{\partial T_k} & = \frac{1}{1 - \rho} T_k^{\frac{1}{1 -\rho} - 1} \sum_{k}^{} (T_c t_{ck} Z_c^{- \theta})^{\frac{1}{1 - \rho}} \\
                                            & =\frac{1}{1 - \rho} \frac{1}{T_k} \lambda_k
\end{align*}

Consider the first term in the derivative:

\begin{align*}
    \frac{\partial \left( \sum_{k}^{} (T_c T_k t_{ck})^{\frac{1}{1 - \rho}} \lambda_k^{- \rho} \right)}{\partial T_k} \frac{T_k}{\sum_{k}^{} T_{ck}^{\frac{1}{1 - \rho}} \lambda_k^{- \rho}} & = \frac{T_k^{\frac{1}{1 - \rho}} (T_c t_{ck})^{\frac{1}{1 - \rho}} (- \rho) \lambda_k^{- \rho - 1} \left( \frac{1}{1 - \rho} \right) \frac{1}{T_k} \lambda_k + \lambda_k^{- \rho} \left( \frac{1}{1 - \rho} \right) \frac{1}{T_k} T_{ck}^{\frac{1}{1 - \rho}}}{\sum_{k}^{} T_{ck}^{\frac{1}{1 - \rho}} \lambda_k^{- \rho}} \\
                                                                                                                                                                                             & = \phi_{ck}
\end{align*}

Consider the second term in the derivative:

\begin{align*}
    - \frac{\partial \left( \sum_{k}^{} \lambda_k^{1 - \rho} \right)}{\partial T_k} \frac{T_k}{\sum_{k}^{} \lambda_k^{1 - \rho}} & = - \frac{(1 - \rho) \lambda_k^{- \rho} \left( \frac{1}{1 - \rho} \right) \frac{1}{T_k} \lambda_k}{\sum_{k}^{} \lambda_k^{1 - \rho}} \times T_k \\
                                                                                                                                 & = - \omega_k
\end{align*}

This will give us:

\begin{align*}
    \frac{\partial \ln \pi_c}{\partial \ln T_k} = \phi_{ck} - \omega_k
\end{align*}

\end{document}