\documentclass[10pt]{article}
\usepackage[utf8]{inputenc}
\usepackage[shortlabels]{enumitem}
\usepackage[margin=1in]{geometry}
\usepackage{setspace}
\onehalfspacing
\setlength{\parskip}{1em}
\setlength{\parindent}{0pt}
\usepackage{natbib}
\bibliographystyle{ecta}
\usepackage[T1]{fontenc}
\usepackage{titling}
\usepackage{amsmath}
\setlength{\droptitle}{-7em}
\addtolength\abovedisplayskip{-3in}
\addtolength\belowdisplayskip{-3in}

\title{City Location Choice and Productivity}
\preauthor{}
\postauthor{}
\author{Tan Sein Jone}
\predate{}
\postdate{}
\date{}
%\Today

\begin{document}
\doublespacing
\maketitle

\section{Joint distribution of productivity across cities}

\begin{align}
    P[Z_c \leq z]                    & = e^{-G^c T_c z^{-\theta}}                                                            \\
    P[Z_1 \leq z, \dots, Z_c \leq z] & = exp[-\sum_{c = 1}^{N}(G^c T_c Z_{c}^{-\theta})^{\frac{1}{1 - \sigma}}]^{1 - \sigma}
\end{align}

$G^c$ is the tail dependence correlation funciton. $\sigma$ determines the substitutability between cities. $T_c$ is the scale parameter that can be subtituted for amenities later on. $\theta$ is the shapre parameter.

\section{Tail dependence correlation function}

\begin{align}
    G^c(x_1, \dots, x_c) = \sum_{k = 1}^{K}[\sum_{s = 1}^{N}(w_{sk}x_{sc})^{\frac{1}{1 - \rho_k}}]^{1 - \rho_k}
\end{align}

$w_{sk}$ is the weight of technology $k$ for sector $s$ which is common between cities. $\rho_k$ is the substitutability of technologies. $x_{s}^{c}$ is the expenditure in sector $s$ for city $c$, this can be analogous to endowments for each city.

\section{Individual distributions}

\begin{align}
    P[Z_{csk} \leq z] = exp [-((w_{sk} x_{sc})^{\frac{1}{1 - \rho_k}} T_c Z_{c}^{-\theta})^{\frac{1}{1 - \sigma}}]
\end{align}

Specific Fréchet distribution for city $c$, sector $s$ and technology $k$.

\begin{equation}
    \phi_c =
    \begin{pmatrix}
        z_{11} & \cdots & z_{1k} \\
        \vdots & \ddots & \vdots \\
        z_{s1} & \cdots & z_{sk}
    \end{pmatrix}
\end{equation}

$\phi_c$ is the matrix of productivity draws from their respective Frechet distributions for each sector and technology in city $c$.

\section{Individual specific technology endowments}

\begin{equation}
    \omega_p =
    \begin{pmatrix}
        v_{1}  & \\
        \vdots & \\
        v_{k}  &
    \end{pmatrix}
\end{equation}

$\omega_p$ is the vector of technology endowments for each person $p$. For now, each endowment is assumed to be drawn from a normal distribution.

\section{Wage realization}

\begin{align}
    \tilde{w_{cp}} & = \phi_c \omega_p \\
                   & = \begin{pmatrix}
                           w_{1}  & \\
                           \vdots & \\
                           w_{s}  &
                       \end{pmatrix}
\end{align}

Realised wage for each sector in city $c$.

\section{Worker problem}

\begin{align}
    \max_{w} [\tilde{w_1}, \dots, \tilde{w_c} | \omega_p]
\end{align}

\end{document}